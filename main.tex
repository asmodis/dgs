\documentclass[a4paper, 12pt, halfparskip*,titlepage]{scrreprt}


\usepackage[ngerman]{babel}
\usepackage[utf8]{inputenc}
\usepackage[T1]{fontenc}

\usepackage[intlimits]{amsmath}
\usepackage{amssymb,amsthm,enumerate}


\newcommand{\N}{\mathbb{N}} 
\newcommand{\Z}{\mathbb{Z}}
\newcommand{\Q}{\mathbb{Q}}
\newcommand{\R}{\mathbb{R}}
\newcommand{\C}{\mathbb{C}}
\newcommand{\K}{\mathbb{K}}
\newcommand{\Hq}{\mathbb{H}}

\newcommand{\set}[1]{\left\lbrace #1 \right\rbrace}

\DeclareMathOperator{\id}{id}
\DeclareMathOperator{\pr}{pr}
\DeclareMathOperator{\incl}{incl}
\DeclareMathOperator{\sgn}{sgn}

\DeclareMathOperator{\Dom}{Dom}
\DeclareMathOperator{\Img}{Img}
\DeclareMathOperator{\supp}{supp}

\DeclareMathOperator{\Abb}{Abb}
\DeclareMathOperator{\Hom}{Hom}
\DeclareMathOperator{\Mult}{Mult}
\DeclareMathOperator{\Alt}{Alt}
\DeclareMathOperator{\End}{End}
\DeclareMathOperator{\Diff}{Diff}
\DeclareMathOperator{\Aut}{Aut}

\newcommand{\dt}[1][t]{\,\mathrm{d} #1}
\newcommand{\ddt}[1][t]{\frac{\mathrm{d}}{\mathrm{d} #1}}
\newcommand{\deldelxi}[1][x^i]{\frac{\partial}{\partial #1}}
\def\eval#1\at#2{\left.{#1}\right|_{#2}}



\newcounter{propc}
\newenvironment{properties}%
{\begin{list}{(\textit{\roman{propc}})}{%
\setlength\leftmargin{3em}%
\usecounter{propc}}}%
{\end{list}}

\newcounter{statec}
\newenvironment{statements}%
{\begin{list}{(\alph{statec})}{%
\setlength\leftmargin{3em}%
\usecounter{statec}}}%
{\end{list}}

\newenvironment{beweis}{\begin{proof}[Beweis]}{\end{proof}}


\newcommand{\sonst}{\text{sonst}}
\newcommand{\tensor}{\otimes}
\newcommand{\bigtensor}{\bigotimes}
\newcommand{\dotted}[1]{#1 \dots #1}
\newcommand{\tensordots}{\dotted{\tensor}}

\newtheorem{proposition}{Proposition}
\newtheorem{satz}{Satz}
\newtheorem{lemma}{Lemma}

\theoremstyle{definition}
\newtheorem*{definition}{Definition}

\theoremstyle{remark}
\newtheorem*{bemerkung}{Bemerkung}
\newtheorem*{kommentar}{Kommentar}
\newtheorem*{notation}{Notation}

\newcommand{\VR}{Vektorraum}
\newcommand{\VB}{Vektorbündel}
\newcommand{\Mfk}{Mannigfaltigkeit}


\setcounter{secnumdepth}{1}

\begin{document}

\title{Differentialgeometrie}
\author{{\tt c\hspace{-1.8pt}izn\hspace{-1.3pt}n0c\hspace{-1.8pt}l:z}}
\maketitle


\pagestyle{empty}
\tableofcontents
\newpage
\pagestyle{plain}

\chapter{Grundlagen}
\label{chap:basics}

Bevor man zum Geometrieteil der Differentialgeometrie kommen kann,
muss man den notwendigen Begriffsapparat aufbauen. Dieser Teil ist
im Prinzip nicht schwierig, er ist durch die vielen verschiedenen, zum
Teil neuen, Konzepte und Begriffe relativ umfangreich.


\section{Vorbemerkungen}
\label{sec:vorbem}

TODO

%%% Local Variables: 
%%% mode: latex
%%% TeX-master: "main"
%%% End: 

\section{Mannigfaltigkeiten}
\label{sec:mfk}

\begin{definition}
  Sei $M$ eine nichtleere Menge und $m\in\N_0$. Ein Tupel $(U,\phi)$ heißt \emph{Karte auf
    $M$}, falls
  \begin{properties}
  \item $U \subset M$,
  \item $\phi\colon U \to \R^m$ eine injektive Abbildung ist und
  \item $\phi(U)$ eine offene Teilmenge des $\R^m$ ist.
  \end{properties}
  Man nennt dann $U$ \emph{Kartengebiet} und $\phi$
  \emph{Koordinatenabbildung} ($\phi^{-1}$ wird auch \emph{lokale
  Parametrisierung} genannt). Ist $p\in U \subset M$, so sagt man
auch, dass $(U,\phi)$ eine Karte \emph{um $p$} ist und falls $\phi(p)
= 0$, so sagt man, dass $(U,\phi)$ eine \emph{um $p$ zentrierte} Karte ist. 

  Eine Familie von Karten $A = \set{(U_i,\phi_i)}_{i\in  I}$ heißt
  \emph{(glatter) Atlas auf $M$}, falls
  \begin{properties}
  \item $\exists m \in \N_0\ \forall i\in I\colon \phi_i(U_i) \subset \R^m$,
  \item $\bigcup_{i\in I} U_i = M$,
  \item $\phi_i(U_i\cap U_j), \phi_j(U_i\cap U_j)$ sind offen und
  \item $\phi_i\circ \phi_j^{-1} \colon \phi_j(U_i\cap U_j) \to
    \phi_i(U_i\cap U_j)$ ist $C^\infty$.
  \end{properties}

  Ein Atlas $A$ heißt \emph{maximal}, falls für jede Karte $(U,\phi)$,
  für die $A\cup \set{(U,\phi)}$ ein glatter Atlas ist, bereits gilt, dass
  $(U,\phi) \in A$. Ein maximaler glatter Atlas heißt auch
  \emph{glatte Struktur}.

  Ein Tupel $(M,D)$ heißt \emph{(glatte) Mannigfaltigkeit}, falls $D$
  eine glatte Struktur ist. Eine Karte $(U,\phi)$ heißt \emph{zulässig},
  falls $(U,\phi)\in D$. Die \emph{Dimension} $\dim M$ einer
  Mannigfaltigkeit ist definiert als $\dim \phi(U)$ für eine zulässige
  Karte (nach Definition eines Atlanten ist die Dimension wohldefiniert). 
\end{definition}

\begin{proposition}\label{prop:basic:atlasbestimmtstruktur}
  Sei $A$ ein Atlas. Dann gibt es genau eine glatte Struktur $D$,
  sodass $A\subseteq D$.

  \textsc{Notation:} Für dieses $D$ schreiben wir auch $[A]$.
\end{proposition}

\begin{proposition}
  Sei $(M,D)$  eine Mannigfaltigkeit und $p\in M$. Dann gibt es für
  alle $0 < r \leq \infty$ eine zulässige um $p$ zentrierte Karte
  $(U,\phi)$ mit $\phi(U) = B_r(0)$.
\end{proposition}

\begin{definition}
  Sei $A$ ein Atlas. Die Initialtopologie bezüglich $A$ heißt
  \emph{Atlastopologie}, in Zeichen $\tau(A)$. Ist $A$ maximal so heißt die Atlastopologie
  auch \emph{Mannigfaltigkeitstopologie}.

  Eine Mannigfaltigkeit $(M,D)$ ist immer mit der
  Mannigfaltigkeitstopologie versehen.
\end{definition}

\begin{proposition}
  Seien $A,B$ Atlanten, sodass $A\cup B$ wieder ein Atlas ist. Dann
  gilt $\tau(A) = \tau(B)$. Insbesondere gilt also $\tau(A) =
  \tau([A])$, die Mannigfaltigkeitstopologie wird also durch eine
  geeignete Atlastopologie festgelegt.
\end{proposition}

\begin{proposition}
  Sei $(M,D)$ eine Mannigfaltigkeit und $\tau$ eine Topologie auf
  $M$. Dann gilt $\tau = \tau(D)$, falls für jede zulässige Karte
  $(U,\phi)$ gilt, dass
  \begin{properties}
  \item $U$ eine offene Menge bezüglich $\tau$ ist und
  \item $\phi \colon U \to \phi(U)$ ein Homeomorphismus bezüglich
    $\tau$ ist.
  \end{properties}
\end{proposition}

\begin{proposition}
  Sei $(M,D)$ eine Mannigfaltigkeit. Dann hat $M$ abzählbare
  Topologie, falls es einen abzählbaren Atlas $A$ mit $D=[A]$ gibt.

  $M$ ist hausdorffsch, falls es für alle Punkte $p,q\in M$ entweder
  eine zulässige Karte $(U,\phi)$ gibt mit $p,q\in U$ oder es zwei
  zulässige Karten $(U,\phi),(V,\psi)$ gibt mit $p\in U$, $q\in V$ und
  $U\cap V = \emptyset$.
\end{proposition}

\begin{proposition}
  Sei $(M,D)$ eine Mannigfaltigkeit. Hat $M$ abzählbare Topologie und
  ist hausdorffsch, so ist $M$ parakompakt.
\end{proposition}

\textbf{VEREINBARUNG:} Ab jetzt gilt immer, dass die
Mannigfaltigkeitstopologie abzählbar und hausdorffsch ist. Wenn von
Karten die Rede ist, sind immer zulässige Karten gemeint.

Wir werden in Zukunft die glatte Struktur nicht mehr unbedingt mit
angeben und erlauben uns, die Dimension einer Mannigfaltigkeit als
Hochzahl anzugeben, das heißt $M^m$ bezeichnet eine Mannigfaltigkeit
der Dimension $m$ (außer es ist aus dem Kontext klar, dass ein
$m$-faches kartesisches Produkt gemeint ist).

%%% Local Variables: 
%%% mode: latex
%%% TeX-master: "main"
%%% End: 

\section{Differenzierbare Abbildungen}
\label{sec:diffabb}

\begin{definition}
  Seien $M,N$ Mannigfaltigkeiten, $O\subset M$ offen und
  $p\in O$. Eine Abbildung $f\colon O \to N$
  heißt \emph{$k$-fach differenzierbar in $p$} (man sagt auch
  \emph{ist $C^k$ in $p$}), falls $f$ stetig ist und für alle Karten $(U,\phi)$ um $p$
  und alle Karten $(V,\psi)$ um $f(p)$ gilt, dass $\psi\circ
  f\circ\phi^{-1} \colon phi(U\cap f^{-1}(V)) \to \psi(f(U)\cap V)$
  $C^k$ in $\phi(p)$ ist. Die Abbildung $f$ ist
  \emph{$C^k$ auf $O$}, falls für alle $p\in O$ gilt, dass $f$ $C^k$
  in $p$ ist. Wir bezeichnen mit $C^k(O,N)$ die Abbildungen $f\colon O
  \to N$ die auf ganz $O$ $C^k$ sind und speziell sei $C^k(O) :=
  C^k(O,\R)$. 

  Eine Abbildung $f\colon O\to N$ heißt \emph{$C^k$-Diffeomorphismus},
  falls $f$ bijektiv ist, $f$ $C^k$ auf $O$ ist und $f^{-1}$ $C^k$ auf
  $N$ ist.

  Zwei Mannigfaltigkeiten heißen \emph{$C^k$-diffeomorph}, falls es
  einen $C^k$-Diffeomorphismus zwischen ihnen gibt.
\end{definition}

\textbf{VEREINBARUNG:} Differenzierbar oder glatt heißt bei uns immer
$C^\infty$. Ein Diffeomorphismus ist glatt (falls nicht anders erwähnt).

\begin{proposition}
  Seien $M,N$ Mannigfaltigkeiten, $p\in M$ und $f\colon M \to
  N$ stetig. Damit $f$ $C^k$ in $p$ ist genügt es, dass es eine Karte
  $(U,\phi)$ um $p$ und eine Karte $(V,\psi)$ um $f(p)$ gibt, sodass
  $\psi\circ f\circ \phi^{-1}$ $C^k$ in $\phi(p)$ ist. 

  Insbesondere
  ist also $f\colon M \to \R$ $C^k$ in $p$, falls es eine Karte
  $(U,\phi)$ um $p$ gibt, sodass $f\circ \phi^{-1}$ $C^k$ um $\phi(p)$ ist.
\end{proposition}

\begin{proposition}
  Sei $M$ eine Mannigfaltigkeit und $(U,\phi)$ eine Karte. Dann ist
  $\phi\colon U \to \phi(U)$ ein Diffeomorphismus.
\end{proposition}

\begin{proposition}
  Seien $M,N,P$ Mannigfaltigkeiten, $f\colon M \to N$ und $g\colon
  N\to P$ Abbilungen. Ist $f$ $C^k$ in $p$ und $g$ ist $C^k$ in
  $f(p)$, so ist $g\circ f$ $C^k$ in $p$.
\end{proposition}

\begin{definition}
 Sei $M$ eine Mannigfaltigkeit. Eine
  \emph{Zerlegung der Eins auf $M$} ist eine
  Familie $\set{\epsilon_i}_{i\in I}$ von Abbildungen, sodass
  \begin{properties}
  \item Die Menge $\set{\supp\epsilon_i}_{i\in I}$ ist lokal endlich,
  \item $\epsilon_i \colon M \to \R$ ist glatt,
  \item $0\leq \epsilon_i \leq 1$ und
  \item $\sum_{i\in I}\epsilon_i(p) = 1$ für alle $p\in M$. 
  \end{properties}

  Ist $\mathcal{U_j}_{j\in J}$ eine offene Überdeckung von $M$, so
  heißt eine Zerlegung der Eins $\set{\epsilon_i}_{i\in I}$ der
  Überdeckung $\mathcal{U_j}_{j\in J}$ \emph{untergeordnet}, falls es
  für alle $i\in I$ ein $j\in J$ gibt, sodass $\supp\epsilon_i \subset
  U_j$.
\end{definition}

\begin{proposition}
  Sei $M$ eine Mannigfaltigkeit und $\mathcal{U_j}_{j\in J}$ eine
  offene Überdeckung von $M$. Dann gibt es eine abzählbare Zerlegung
  der Eins $\set{\epsilon_i}_{i\in I}$, die $\mathcal{U_j}_{j\in J}$ untergeorddnet ist, sodass
  $\supp\epsilon_i$ kompakt ist.

  Verzichtet man auf die Kompaktheit des Trägers, so gibt es eine
  Zerlegung der Eins $\set{\epsilon_j}_{j\in J}$ mit $\supp\epsilon_j
  \subset U_j$ (gleicher Index!) wobei höchstens abzählbar viele
  $\epsilon_j$ nicht identisch verschwinden.
\end{proposition}

\begin{korollar}
  Sei $M$ eine Mannigfaltigkeit, $O\subset M$ offen und $A \subset O$
  abgeschlossen. Dann gibt es eine glatte Abbildung $h\colon M \to
  \R$, sodass $h|_A = 1$ und $h|_{M\setminus U} = 0$ und $0 \leq h
  \leq 1$.

  Eine solche Abbildung heißt \emph{Hutfunktion}.
\end{korollar}

%%% Local Variables: 
%%% mode: latex
%%% TeX-master: "main"
%%% End: 

\section{Der Tangentialraum an einen Punkt}
\label{sec:tpm}

\begin{definition}
  Sei $M$ eine Mannigfaltigkeit und $p\in M$. Wir definieren für $f,g
  \in C^\infty(M)$ die Relation
  \begin{equation*}
    f \sim_p g \iff \exists U\text{ Umgebung von $p$ mit } f|_U = g|_U
  \end{equation*}
  und bezeichnen mit $f_p := \set{g\in C^\infty(M) \mid g\sim_p f}$
  den sogenannten \emph{glatten Funktionskeim von $f$ in $p$} und mit
  $C^\infty(p)$ die Menge der glatten Funktionskeime in $p$.
\end{definition}

\begin{proposition}
  Sei $M$ eine Mannigfaltigkeit und $p\in M$. Dann gilt
  \begin{statements}
  \item $\sim_p$ ist eine Äquivalenzrelation.
  \item $C^\infty(p)$ ist eine kommutative $\R$-Algebra.
  \item Die Abbildung $C^\infty(p) \ni f_p \mapsto f(p) \in \R$ ist
    ein wohldefinierer $\R$-Algebrahomomorphismus. 
  \end{statements}
\end{proposition}

\begin{definition}
  Seien $A,B$ $\R$-Algebren und $\rho \colon A \to B$ ein
  Algebrahomomorphismus. Eine lineare Abbildung $\xi \colon A \to B$
  heißt \emph{lineare Derivation entlang $\rho$}, falls
  \begin{equation*}
    \xi(ab) = \xi(a)\rho(b) + \rho(a)\xi(b)
  \end{equation*}
  gilt.
\end{definition}

\begin{definition}
  Sei $M$ eine Mannigfaltigkeit und $p\in M$. Dann ist der
  \emph{Tangentialraum an $p$} definiert als
  \begin{equation*}
    T_pM := \Der_p(M) := \set{ \xi \colon C^{infty}(p) \to \R\mid
      \xi\text{ ist eine lineare Derivation entlang $f_p \mapsto f(p)$}}
  \end{equation*}
  Ein $\xi \in T_pM$ heißt \emph{Tangentialvektor}.

  Ist $f_p\in C^\infty(p)$ so heißt $\xi(f_p)$ die
  \emph{Richtungsableitung von $f$ in Richtung $\xi$}.
\end{definition}

\begin{proposition}
  Sei $M$ eine Mannigfaltigkeit, $p\in M$, $U$ eine Umgebung von $p$
  und $f\in C^\infty(U)$. Sei weiterhin $h$ eine Hutfunktion mit $p
  \in \supp h \subset U$. Für $\xi \in T_pM$ ist dann $\xi(f) := \xi((hf)_p)$
  wohldefiniert, das heißt unabhängig von $h$.

  \textsc{Notation:} Ist $f\in C^\infty(U)$ so bezeichne $f_p$ den
  Keim einer beliebigen Fortsetzung von $f$ auf $M$. Nach dieser
  Proposition ist dies wohldefiniert. Für $\xi \in T_pM$ bezeichne $\xi(f) := \xi(f_p)$.
\end{proposition}

\begin{definition}
  Sei $M$ eine Mannigfaltigkeit, $p\in M$ und $(U,x)$ eine Karte um
  $p$. Dann heißt
  \begin{equation*}
    \eval \deldelxi \at{p}\colon C^\infty(p) \to \R, f_p \mapsto \eval
    \deldelxi f\circ x^{-1}\at{x(p)}
  \end{equation*}
  der \emph{$i$-te Koordinatenvektor} für $i = 1,\dots,\dim M$.
\end{definition}

\begin{proposition}
  Sei $M$ eine Mannigfaltigkeit, $p\in M$ und $\xi \in T_pM$. Ist
  $c\colon p \mapsto c \in C^\infty(M)$ eine konstante Abbildung, so
  gilt $\xi(c) = 0$.

  Sind $f,g\in C^\infty(M)$ mit $f(p) = g(p) = 0$, so gilt $\xi(fg) = 0$.
\end{proposition}

\begin{lemma}
  Sei $U\subset \R^n$ ein Sterngebiet mit Mittelpunkt $x_0$ und
  $f\colon U \to \R$ sei $C^\infty$. Dann gibt es glatte Abbildungen
  $g_i \colon U \to \R$, sodass
  \begin{equation*}
    f(x) = f(x_0) + \sum_{i=1}^n g_i(x) \pr^i(x)
  \end{equation*}
  für alle $x\in U$, wobei $\pr^i$ die kanonische Projektion auf die
  $i$-te Komponente ist.
\end{lemma}

\begin{proposition}
  Sei $M^m$ eine Mannigfaltigkeit, $p\in M$ und $(U,x)$ eine Karte um
  $p$. Dann bilden die Koordinatenvektoren $\kvecxi[1],\dots,\kvecxi[m]$
  eine Basis von $T_pM$. Insbesondere ist also $\dim T_pM = \dim M$.
\end{proposition}

\begin{proposition}
  Sei $M^m$ eine Mannigfaltigkeit und $(U,x),(V,y)$ zwei Karten um $p\in
  M$. Sei $\xi \in T_pM$ mit
  \begin{equation*}
    \xi = \sum_{i=1}^m \eta^i \kvecxi = \sum_{i=1}^m \mu^i \kvecyi
  \end{equation*}
  und sei $\eta = (\eta^1,\dots,\eta^m)$, $\mu =
  (\mu^1,\dots,\mu^m)$. Dann gilt
  \begin{equation*}
    \eta = \Jac_{y(p)}(x\circ y^{-1})\mu
  \end{equation*}
\end{proposition}

\begin{definition}
  Sei $M$ eine Mannigfaltigkeit und sei $I\subset \R$ ein offenes Intervall. Eine glatte
  Abbildung $\alpha \colon I \to M$ heißt \emph{glatter Weg (glatte
    Kurve)} und falls $0\in I$ und $\alpha(0) = p$, so heißt $\alpha$
  auch \emph{Kurve (Weg) durch $p$}.

  Man definiert den \emph{Tangentenvektor von $\alpha$ zur Zeit $s$} durch
  \begin{equation*}
    \dot\alpha(s)(f) := \eval\ddt f\circ\alpha(t)\at{t=s} 
  \end{equation*}
  für alle $f\in C^\infty(\alpha(s))$.
\end{definition}

\begin{proposition}
  Sei $M$ eine Mannigfaltigkeit und $\alpha \colon I \to M$ eine
  glatte Kurve. Dann gilt $\dot\alpha(s) \in T_{\alpha(s)}M$.
\end{proposition}

\begin{proposition}
  Sei $M$ eine Mannigfaltigkeit. Ist $M$ ein Vektorraum, so gibt es
  für alle $p\in M$
  einen kanonischen Isomorphismus zwischen $M$ und $T_pM$.
\end{proposition}

%%% Local Variables: 
%%% mode: latex
%%% TeX-master: "main"
%%% End: 

\section{Das Differential einer Abbildung in einem Punkt}
\label{sec:differential}

\begin{definition}
  Seien $M,N$ Mannigfaltigkeiten, $f\colon M \to N$ glatt in $p\in
  M$. Das \emph{(totale) Differential von $f$ in $p$} ist
  \begin{equation*}
    d_pf \colon T_pM \to T_{f(p)}N, \xi \mapsto d_pf(\xi)
  \end{equation*}
  wobei $d_pf(\xi)$ für $g\in C^\infty(N)$ wie folgt definiert ist
  \begin{equation*}
    d_pf(\xi)(g) := \xi(g\circ f)
  \end{equation*}
\end{definition}

\begin{proposition}
  Seien $M,N$ Mannigfaltigkeiten, $f\colon M \to N$ glatt in $p\in
  M$. Dann ist das Differential $d_pf$ eine wohldefinierte lineare Abbildung.
\end{proposition}

\begin{definition}
  Seien $M,N$ Mannigfaltigkeiten, $f\colon M \to N$ glatt in $p\in
  M$. Dann ist der \emph{Rang von $f$ in $p$}, $\rnk_p f$, wie folgt definiert
  \begin{equation*}
    \rnk_p f := \dim d_pf(T_pM),
  \end{equation*}
  also als Rang der linearen Abbildung $d_pf$. Man sagt, $f$ hat
  \emph{vollen Rang in $p$}, falls $\rnk_p f = \min(\dim M,\dim N)$.

  Hat die Abbildung $f$ vollen Rang, so heißt sie
  \begin{statements}
  \item \emph{Submersion in $p$}, falls  $\dim M \geq \dim N$,
  \item \emph{Immersion in $p$}, falls $\dim M \leq \dim N$ und
  \item \emph{Bimersion in $p$}, falls $\dim M = \dim N$.
  \end{statements}

  Eine glatte Abbildung $f\colon M \to N$ heißt \emph{Einbettung},
  falls $f$ eine injektive Immersion ist und $f\colon M \to f(M)$ ein
  Homeomorphismus ist wenn man $f(M)$ mit der Teilraumtopologie versieht.
\end{definition}

\begin{proposition}
  Seien $M,N$ Mannigfaltigkeiten, $f\colon M \to N$ glatt in $p\in
  M$. Dann ist $f$ genau dann eine
  \begin{statements}
  \item Submersion, falls $d_pf$ surjektiv ist.
  \item Immersion, falls $d_pf$ injektiv ist.
  \item Bimersion, falls $d_pf$ bijektiv ist.
  \end{statements}
\end{proposition}

\begin{proposition}
  Seien $M,N$ Mannigfaltigkeiten, $f\colon M \to N$ glatt. Dann ist
  $\rnk f$ unterhalbsstetig, das heißt ist $\rnk_p f = k$, so gibt es
  eine Umgebung $U$ von $p$, sodass $\rnk_q f \geq k$ für alle $q\in
  U$.

  Insbesondere haben Submersionen, Immersionen und Bimersionen einen
  lokal konstanten Rang.
\end{proposition}

\begin{satz}[Satz vom konstanten Rang]
  Seien $M^m,N^n$ Mannigfaltigkeiten, $f\colon M \to N$ glatt. Sei $p\in
  M$ und $f$ habe lokal um $p$ konstanten Rang, das heißt es gibt eine
  Umgebung $U$ um $p$, sodass $\rnk_qf = \rnk_pf$ für alle $q\in
  U$. Dann gibt es eine Karte $(U,x=(x^1,\dots,x^m))$ um $p$ und eine
  Karte $(V,y=(y^1,\dots,y^n))$ um $f(p)$, sodass $y\circ f = (x^1,\dots,x^k,0,\dots,0)$.
\end{satz}

\begin{korollar}[Normalformensatz]
  Seien $M^m,N^n$ Mannigfaltigkeiten, $f\colon M \to N$ glatt. Dann gibt
  es genau dann eine Karte $(U,x)$ um $p\in M$ und eine Karte $(V,y)$ um $f(p)$, sodass
  \begin{statements}
  \item $y\circ f\circ x^{-1} = \pr$, falls $f$ eine Submersion (und
    damit $m\geq n$) ist
    (wobei $\pr$ die Projektion $\R^m\to\R^n, (x^1,\dots,x^m) \mapsto
    (x^1,\dots,x^n)$ bezeichnet).
  \item $y\circ f\circ x^{-1} = \incl$, falls $f$ eine Immersion (und
    damit $m\leq n$) ist
    (wobei $\incl$ die Projektion $\R^m\to\R^n, (x^1,\dots,x^m) \mapsto
    (x^1,\dots,x^m,0,\dots,0)$ bezeichnet).
  \item $y\circ f\circ x^{-1} = \id$, falls $f$ eine Bimersion (und
    damit $m = n$) ist.
  \end{statements}
\end{korollar}

\begin{korollar}[Umkehrsatz]
  Seien $M,N$ Mannigfaltigkeiten, $f\colon M \to N$ glatt. Dann ist
  $f$ genau dann eine Bimersion in $p$, falls $f$ ein lokaler
  Diffeomorphismus um $p$ ist, das heißt es gibt eine Umgebung von
  $p$, sodass $f|_U\colon U \to f(U)$ ein Diffeomorphismus ist. 
\end{korollar}

\begin{proposition}
  Seien $M,N$ Mannigfaltigkeiten, $f\colon M \to N$ glatt mit vollem
  Rang. Ist $f$ injektiv, so ist $f$ eine Immersion.
\end{proposition}

\begin{proposition}
  Seien $M,N$ Mannigfaltigkeiten und $f\colon M \to N$ eine
  Submersion. Dann ist $f$ offen, das heißt für jede offene Menge
  $U\subseteq M$ gilt, dass $f(U) \subseteq N$ offen ist.
\end{proposition}

\begin{proposition}
  Seien $M,N,P$ Mannigfaltigkeiten und $f\colon M\to N$, $g\colon N
  \to P$ glatte Abbildungen. Dann ist $d_p g\circ f = d_{f(p)}g \circ d_pf$.
\end{proposition}

\begin{proposition}
  Seien $M,N$ Mannigfaltigkeiten, $M$ zusammenhängend und $f\colon M
  \to N$ glatt. Gilt für
  alle $p\in M$, dass $d_pf = 0$, so ist $f$ konstant. 
\end{proposition}

%%% Local Variables: 
%%% mode: latex
%%% TeX-master: "main"
%%% End: 

\section{Produkt- und Untermannigfaltigkeiten}
\label{sec:prumfk}

\begin{proposition}
  Seien $M_1^{m_1},\dots,M_n^{m_n}$ Mannigfaltigkeiten. Dann gibt es
  eine eindeutig bestimmte glatte Struktur auf $M :=
  M_1\times\dots\times M_n$, sodass die Projektionen $\pi_i \colon M
  \to M_i$ Submersionen sind und $\dim M = \sum_{i=1}^n m_i$.
\end{proposition}

\begin{definition}
  Seien $M_1^{m_1},\dots,M_n^{m_n}$ Mannigfaltigkeiten. Die obige
  eindeutig bestimmte glatte  Struktur heißt
  \emph{Produktmannigfaltigkeitsstruktur}. Wir vereinbaren, dass wir
  ein Produkt von Mannigfaltigkeiten mit dieser Struktur versehen,
  sofern nichts anderes erwähnt wird.
\end{definition}

\begin{proposition}
  Seien $M_1,\dots,M_n,P$ Mannigfaltigkeiten und $M :=
  M_1\times\dots\times M_n$. Dann ist $f\colon P \to M$ genau dann
  $C^k$, falls $\pi_i\circ f \colon P \to M_i$ für $i=1,\dots,n$ $C^k$
  ist, wobei $\pi_i$ die kanonische Projektion bezeichnet.
\end{proposition}

\begin{definition}
  Sei $M^m$ eine Mannigfaltigkeit und $N \subseteq M$. 
  Man nennt $N$ \emph{Untermannigfaltigkeit von $M$ der Kodimension
    $k$}, falls es um jeden Punkt $p\in N$ eine Karte $(U,x)$ gibt, sodass
  \begin{equation*}
    x(U\cap N) = \R^{m-k}\times\set{0}^{k} \cap x(U).
  \end{equation*}
  Eine solche Karte heißt \emph{angepasst an $N$}.
\end{definition}

\begin{proposition}
  Sei $M^m$ eine Mannigfaltigkeit und $N\subset M$ eine
  Untermannigfaltigkeit der Kodimension $k$. Dann gibt es eine
  eindeutige glatte Struktur auf $N$, sodass $N$ eine
  $m-k$-dimensionale Mannigfaltigkeit wird und die kanonische
  Inklusion $\incl\colon N \to M$ eine Einbettung ist.
\end{proposition}

\begin{korollar}
  Sei $M$ eine Mannigfaltigkeit und $O\subseteq M$ offen. Dann gibt es
  eine eindeutig bestimmte glatte Struktur auf $O$, sodass
  $\incl\colon O \to M$ eine Einbettung ist.

  Wir werden offene Teilmengen mit dieser Struktur versehen, wenn
  nichts anderes erwähnt wird.
\end{korollar}

\begin{satz}
  Sei $M^m$ eine Mannigfaltigkeit und $N\subseteq M$. Dann sind äquivalent
  \begin{statements}
  \item $N$ ist eine Untermannigfaltigkeit der Kodimension $k$.
  \item Zu jedem Punkt $p\in N$ gibt es eine in $M$ offene Umgebung
    $U\subset M$ von $p$ und einen Diffeomorphismus $\Phi\colon U \to
    \Phi(U) \subseteq \R^m$, sodass $\Phi(U\cap N) = \Phi(U)\cap
    \R^{m-k}\times\set{0}^k$.
  \item Zu jedem Punkt $p\in N$ gibt es eine in $M$ offene Umgebung
    $U\subset M$ von $p$ und eine Submersion $g\colon U \to \R^k$,
    sodass $U\cap N = g^{-1}(\set{0})$. Man nennt $g$ auch
    \emph{lokale Beschreibung von $N$ um $p$}.
  \item Zu jedem Punkt $p\in N$ gibt es eine in $M$ offene Umgebung
    $U\subset M$ von $p$, eine Umgebung $O\subset \R^{m-k}$ der $0$
    und eine Immersion $\phi\colon O \to M$,
    sodass $\phi(O) = U\cap N$ und $\phi(0) = p$. Man nennt $\phi$
    auch \emph{lokale Parametrisierung um $p$.}
  \end{statements}
\end{satz}

\begin{proposition}
  Seien $M^m,N^n$ Mannigfaltigkeiten und $f\colon M \to N$ glatt. Sei
  $q\in N$ und $P := f^{-1}(\set{q})$ nichtleer. Ist $f$ eine
  Submersion auf $P$,, so ist $P$ eine Untermannigfaltigkeit von $M$
  der Kodimension $n$.
\end{proposition}

\begin{proposition}
  Seien $M,P$ Mannigfaltigkeiten und $f\colon M\to P$ glatt.
  \begin{statements}
  \item Ist $N\subset M$ eine Untermannigfaltigkeit, dann ist auch
    $f|_N\colon N \to P$ glatt.
  \item Ist $N\subset P$ eine Untermannigfaltigkeit und ist
    $f(M)\subseteq N$, dann ist auch
    $f\colon M \to N$ glatt.
  \end{statements}
\end{proposition}

\begin{proposition}
  Sei $M$ eine Mannigfaltigkeit und $N\subseteq M$ eine
  Untermannigfaltigkeit. Dann ist für $p\in N$ die Abbildung $d_p\incl
  \colon T_pN \to T_pM$ injektiv. Man kann also kanonisch $T_pN$ als
  Untervektorraum von $T_pM$ auffassen.

  Ist $g$ eine lokale Beschreibung um $p$ und $\phi$ eine lokale
  Parametrisierung um $p$ mit $\phi(0) = p$, dann gilt
  \begin{equation*}
    T_pN = \ker d_p g = \Img d_0\phi
  \end{equation*}
\end{proposition}

\begin{definition}
  Seien $M,N$ Mannigfaltigkeiten und $f\colon M \to N$ glatt. Falls
  $f$ eine Immersion ist, so heißt $f\colon M\to N$ (beziehungsweise
  $f(M)$) \emph{immersierte Untermannigfaltigkeit}. Falls $f$ eine
  Einbettung ist, so heißt $f\colon M \to N$ (beziehungsweise $f(M)$)
  \emph{eingebettete Untermannigfaltigkeit}.
\end{definition}

\begin{proposition}
  Seien $M,N$ Mannigfaltigkeiten und $f\colon M\to N$ eine Einbettung. Dann ist
  $f(M)$ eine Untermannigfaltigkeit von $N$ mit Kodimension $\dim N -
  \dim M$. Ist umgekehrt $M\subset N$ eine Untermannigfaltigkeit, so
  ist $\incl\colon M \to N$ eine eingebettete Untermannigfaltigkeit.

  \textsc{Sprechweise:} Wir werden daher in Zukunft nicht mehr
  zwischen Untermannigfaltigkeiten und eingebettetn
  Untermannigfaltigkeiten unterscheiden.
\end{proposition}

%%% Local Variables: 
%%% mode: latex
%%% TeX-master: "main"
%%% End: 

\section{Vektorbündel}
\label{sec:faserbl}

\begin{definition}
  Seien $E,M$ Mannigfaltigkeiten, $r\in\N$, $\K=\R,\C$ und $\pi\colon E \to M$ glatt. Man
  nennt $(E,M,\pi)$ \emph{$\K$-Vektorbündel vom Rang $r$ über $M$}, falls
  \begin{properties}
  \item $\pi\colon E \to M$ eine surjektive Submersion ist und
  \item für jeden Punkt $p\in M$ existiert eine Umgebung $U$ von $p$
    und ein Diffeomorphismus $\Phi = (\Phi_1,\Phi_2) \colon \pi^{-1}(U) \to U\times \K^r$,
    sodass $\Phi_1(e) = \pi(e)$ und $\Phi_p :=
    \Phi_2|_{\pi^{-1}(\set{p})} \colon \pi^{-1}(\set{p}) \to \K^r$
    ein linearer Isomorphismus ist.
  \end{properties}
  \textsc{Schreibweise:} Oft schreiben wir nur $\pi\colon
  E\to M$ oder sogar nur $E$ für ein Vektorbündel. Der Rang eines
  Bündels $E$ wird mit $\rnk E$ bezeichnet. Ist der Körper $\K$ aus
  dem Kontext klar (oder egal), so lassen wwir in weg.

  \textsc{Bezeichnungen:} Ist $\pi\colon E\to M$ ein Vektorbündel mit
  $\rnk E = r$, so heißt $E$ \emph{Totalraum}, $M$ \emph{Basisraum} und
  $\pi$ \emph{Bündelprojektion}. Ein Paar $(U,\Phi)$ wie in (ii)
  heißt \emph{Bündelkarte}, $U$ heißt \emph{lokal trivialisierende
    Umgebung} und $\Phi$ heißt \emph{lokale Trivialisierung}. Ein
  \emph{Bündelatlas} ist eine Familie von Bündelkarten, sodass die
  trivialisierenden Umgebungen ganz $M$ überdecken. Man sagt $E$
  \emph{trivialisiert über $U$}, falls es eine Bündelkarte $(U,\Phi)$ gibt.

  Ist $p$ in $M$ und $U\subset M$ offen, so bezeichnet $E_p :=
  \pi^{-1}(\set{p})$ die \emph{Faser über  $p$} und $E_U :=
  \pi^{-1}(U)$ das \emph{Teilbündel über $U$}.

  Ist $(U,\Phi)$ eine Bündelkarte, so bezeichnen $\pr_1,\pr_2\colon
  U\times \K^r$ die kanonischen Projektionen auf die erste und zweite Komponente.
\end{definition}

\begin{proposition}
  Sei $E\to M$ ein $\K$-Vektorbündel vom Rang $r$. Dann gilt $\dim E =
  \dim M + \dim_\R\K\cdot r$.
\end{proposition}

\begin{proposition}
  Sei $E\to M$ ein Vektorbündel und $U\subset M$ offen. Dann ist das
  Teilbündel $E_U$ ein Vektorbündel über $U$.
\end{proposition}

\begin{proposition}
  Sei $M$ eine Mannigfaltigkeit. Dann ist $\trivbl{\K^r} := M\times \K^r$ zusammen mit
  $\pi(p,f) = p$ ein $\K$-Vektorbündel vom Rang $r$. Man nennt
  $\trivbl{\K^r}$ das \emph{triviale $\K$-Vektorbündel}.
\end{proposition}

\begin{definition}
  Seien $\pi_E\colon E\to M$ und $\pi_F\colon F \to N$ zwei
  $\K$-Vektorbündel und $g\colon M\to N$ glatt. Eine Abbildung
  $G\colon E\to F$ heißt \emph{Bündelmorphismus entlang $g$}, falls
  $g\circ\pi_E = \pi_F\circ G$ und $G_p := G|_{E_p} \colon E_p \to
  F_{g(p)}$ linear ist.

  Ist $g$ ein Diffeomorphismus, so heißt eine bijektive Abbildung $G
  \colon E\to F$ \emph{Bündelisomorphismus
    entlang $g$}, falls $G$ ein Bündelmorphismus entlang $g$ und
  $G^{-1}$ ein Bündelmorphismus entlang $g^{-1}$ ist.

  Ist $M=N$ so heißt ein Bündelmorphismus entlang $\id$ einfach nur
  \emph{Bündel(homo)morphismus} und ein Bündelisomorphismus entlang
  $\id$ heißt \emph{Bündelisomorphismus}.

  Zwei Bündel heißen \emph{isomorph (entlang $g$)}, falls es einen
  Bündelisommorphismus (entlang $g$) gibt. Ein Bündel heißt
  \emph{trivialisierbar}, falls es isomorph zum trivialen Bündel ist.
\end{definition}

\begin{proposition}
  Ein Bündel $E\to M$ ist genau dann trivialisierbar, wenn es eine
  globale Bündelkarte $(M,\Phi)$ gibt.
\end{proposition}

\begin{definition}
  Sei $\pi\colon E\to M$ ein $\K$-Vektorbündel vom Rang $r$ und $(U_1,\Phi_1),
  (U_2,\Phi_2)$ zwei Bündelkarten um $p\in M$. Dann heißt
  \begin{equation*}
    g_{12} \colon U_1\cap U_2 \to \GL(\K^r), p\mapsto \pr_2(\Phi_2\circ \Phi_1^{-1}(p,\cdot))
  \end{equation*}
  der \emph{Bündelübergang von $(U_1,\Phi_1)$ nach $(U_2,\Phi_2)$}.
\end{definition}

\subsection*{Charakterisierung durch Kozykel}

\begin{definition}
  Sei $M$ eine Mannigfaltigkeit und $X$ eine Menge von Abbildungen,
  die abgeschlossen unter Komposition ist.
  Sei weiterhin $\set{U_i}_{i\in I}$ eine offene
  Überdeckung von $M$ und $z = \set{ g_{ij} \colon U_i\cap U_j \to X}$
  eine Familie von Abbildungen. Man nennt $z$ einen
  \emph{$1$-Kozykel mit Werten in $X$}, falls er die \emph{Kozykelbedingung}
  \begin{statements}
  \item $g_{ii} = \id$
  \item $g_{ij} \circ g_{jk} = g_{ik}$ auf $U_i\cap U_j\cap U_k$
  \end{statements}
  erfüllt.

  Eine Familie von Abbildungen $\set{h_i\colon U_i \to X}$ heißt
  \emph{$1$-Korand mit Werten in $X$}, falls die $h_i$ invertierbar
  sind. Zwei Kozykel $\set{g_{ij}}$ und
  $\set{\tilde{g}_{ij}}$ heißen \emph{kohomolog}, falls es einen
  Korand $\set{h_i}$ gibt, sodass $\tilde{g}_{ij} = h_i \circ g_{ij}
  \circ h_j^{-1}$ 
\end{definition}

\begin{proposition}
  Sei $\set{(U_i,\Phi_i)}_{i\in I}$ ein Bündelatlas eines $\K$-Vektorbündels
  $\pi\colon E \to M$ vom Rang $r$. Dann bilden die Bündelübergänge $g_{ij}$ einen
  $1$-Kozykel mit Werten in $\GL(\K^r)$. Man nennt diesen Kozykel auch
  \emph{Übergangskozykel (zur Überdeckung $\set{U_i}$)}.
\end{proposition}

\begin{proposition}
  Sei $M$ eine Mannigfaltigkeit und $\set{U_i}_{i\in I}$ eine offene
  Überdeckung von $M$. Ist $z = \set{g_{ij}}$ ein 1-Kozykel mit Werten
  in $\GL(\K^r)$, so gibt es ein $\K$-Vektorbündel vom Rang $r$,
  welches über $\set{U_i}$ trivialisiert und dort $z$ als
  Übergangskozykel hat.
\end{proposition}

\begin{proposition}
  Seien $E\to M$, $F\to M$ Vektorbündel die beide über der offenen
  Überdeckung $\set{U_i}_{i\in I}$ von $M$ trivialisieren. Dann sind $E$ und $F$ genau
  dann isomorph, wenn die jeweiligen Übergangskozykel kohomolog sind.
\end{proposition}

\begin{satz}
  Sei $M$ eine Mannigfaltigkeit und $\set{U_i}_{i\in I}$ eine offene
  Überdeckung von $M$. Ist $z = \set{g_{ij}}$ ein 1-Kozykel mit Werten
  in $\GL(\K^r)$, so gibt es ein bis auf Bündelisomorphie eindeutiges $\K$-Vektorbündel vom Rang $r$,
  welches über $\set{U_i}$ trivialisiert und dessen Übergangskozykel
  zu $z$ kohomolog ist.
\end{satz}

\subsection*{Rahmen und Schnitte}

\begin{definition}
  Sei $\pi \colon E\to M$ ein Vektorbündel vom Rang $r$. Eine
  Abbildung $s\colon M \to E$ heißt \emph{rauher Schnitt in/von $E$}, falls
  $\pi\circ s 0 \id$. Ein \emph{(glatter) Schnitt in/von $E$} ist ein
  rauher Schnitt, der glatt ist. Die Menge aller glatten Schnitte wird
  mit $\Gamma(E)$ bezeichnet. Man schreibt oft $s_p := s(p)$ für einen
  Schnitt $s\in \Gamma(E)$. Ein \emph{lokaler Schnitt über $U\subset
    M$} ist ein (globaler) Schnitt über $E_U$.

  Man nennt Schnitte $s_1,\dots,s_n$ \emph{linear unabhängig}, falls
  sie dies Punktweise sind. Ein \emph{(lokaler) Rahmen  über $U\subset
    M$} $\vbframe{\Psi}
  = (\Psi_1,\dots,\Psi_r)$ ist ein $r$-Tupel aus lokalen Schnitten
  $\Psi_i \in \Gamma(E_U)$, sodass
  $\vbframe{\Psi}_p = (\Psi_1(p),\dots,\Psi_r(p))$ für jedes $p\in U$
  eine Basis von $E_p$ ist.
\end{definition}

\begin{proposition}
  Für$s,t\in\Gamma(E)$ und $f\inC^\infty(M)$ wird durch $(fs + t)(p)
  := f(p)s(p) + t(p)$ die Menge $\Gamma(E)$ zu einem lokal freien $C^\infty(M)$-Modul.
\end{proposition}

\begin{proposition}
  Ist $E\to M$ ein Vektorbündel, so gibt es eine bijektive
  Korrespondenz zwischen Bündelkarten $(U,\Phi)$ und lokalen Rahmen
  $\vbframe{\Psi}$ über $U$ in dem Sinne, dass es zu jeder Bündelkarte
  einen lokalen Rahmen gibt und umgekehrt.
\end{proposition}

\subsection*{Konstruktion neuer Vektorbündel}



%%% Local Variables: 
%%% mode: latex
%%% TeX-master: "main"
%%% End: 


%
\section{Lineare und Multilineare Algebra}
\label{sec:mla}

\textbf{Vereinbarung:} In diesem Abschnitt sei $\K=\R,\C$ (also ein
vernünftiger Körper) und alle
auftretenden Vektorräume (bis auf die frei erzeugten) seien
endlichdimensionale $\K$-Vektorräume. Insbesondere sollen
$V_1,\dots,V_n, U,V,W$ endlichdimensionale $\K$-Vektorräume bezeichnen

\subsection{Vorbemerkungen (aka Erinnerungen an die Grundvorlesungen)}
\label{sec:vbakaegv}

Wir geben hier eine Sammlung von Begriffen und ihren Eigenschaften,
die man kennen sollte.

\begin{definition}
  Sei $M$ eine nichtleere Menge. Versieht man die Menge
  \begin{equation*}
    F(M) := \set{f \colon M \to \K \mid \supp f\text{ endlich}}
  \end{equation*}
  mit der kanonischen Vektorraumstruktur für $\K$-wertige Abbildungen,
  so heißt $F(M)$ der \emph{frei von $M$ erzeugte Vektorraum}.

  Die kanonische Einbettung von $M$ in $F(M)$ bezeichnen wir mit
  \begin{equation*}
    \delta \colon M \to F(M),\, x \mapsto \delta_x
  \end{equation*}
  wobei $\delta_x(y) := 1$ falls $x=y$ und $0$ sonst.
\end{definition}

\begin{proposition}
  Es gilt
  \begin{statements}
  \item Die Menge $\delta(M)$ bildet eine Basis von $F(M)$.
  \item \textit{(Universelle Eigenschaft von $(F(M),\delta)$)} Ist
    $f\colon M \to U$ eine Abbildung, so gibt es genau eine lineare
    Abbildung $F\colon F(M) \to U$, sodass $f = F\circ\delta$.
  \end{statements}
\end{proposition}

\begin{proposition}[Universelle Eigenschaft des Quotienten]
  Sei $U$ ein Unterraum von $V$. Ist $f\colon
  V \to W$ eine lineare Abbildung, sodass $U \subset \ker f$, dann
  gibt es eine eindeutig bestimmte lineare Abbildung $F\colon V/U \to
  W$, sodass $f = F\circ \pr$, wobei $\pr\colon V \to V/U$ die
  kanonische Projektion bezeichnet.
\end{proposition}

\begin{definition}
  Sei $b$ eine bilineare Abbildung $b\colon V\times W \to U$.

  Man nennt $b$ \emph{nichtentartet}, falls
  \begin{equation*}
    \forall v\in V\colon b(v,w) = 0 \implies w=0\text{ und }
    \forall w\in W\colon b(v,w) = 0 \implies v=0
  \end{equation*}

  Man nennt $b$ eine \emph{Paarung}, falls $U=\K$ und eine
  nichtentartete Paarung heißt auch \emph{Dualität}.
\end{definition}

\begin{proposition}
  Eine Dualität liefert je einen Isomorphismus $\Phi\colon V\to W^*$
  und $\Psi\colon W\to V^*$.
\end{proposition}

\subsection{Tensorprodukt}
\label{sec:tensor}

\begin{definition}
  Seien $V,W$ Vektorräume und bezeichne $I(V\times W)$ den von den Elementen
  \begin{equation*}
    \delta((\lambda v, w)) - \lambda \delta((v,w)),\, 
    \delta((v, \lambda w)) - \lambda \delta((v,w)),\,
    \delta((v + v', w)) - \delta((v,w)) -\delta((v',w)),\,
    \delta((v, w + w')) - \delta((v,w)) -\delta((v,w'))
  \end{equation*}
  erzeugten Vektorunterraum von $F(V\times W)$, wobei $v,v'\in V$,
  $w,w'\in W$ und $\lambda\in\K$. Dann ist das \emph{Tensorprodukt von
    $V$ und $W$ über $\K$} definiert als
  \begin{equation*}
    V\tensor_\K W := V \tensor W := F(V\times W)/I(V\times W).
  \end{equation*}

  Wir bezeichnen mit $\tau\colon V\times W \to V\tensor W, \tau((v,w))
  := v\tensor w := \pr\circ\delta((v,w))$ die kanonische Einbettung
  von $V\times W$ in $V\tensor W$, wobei $\pr\colon F(V\times W) \to
  F(V\times W)/I(V\times W)$ die kanonische Projektion ist.
\end{definition}

\begin{proposition}
  Seien $U,V,W$ Vektorräume. Dann gilt
  \begin{statements}
  \item Die Abbildung $\tau$ ist bilinear.
  \item Die Menge $\tau(V\times W)$ bildet ein Erzeugendensystem von
    $V\tensor W$.
  \item \textit{(Universelle Eigenschaft des Tensorprodukts)} Sei $b\in
  \Mult(V,W;U)$. Dann gibt es eine eindeutig bestimmte lineare
  Abbildung $B\colon V\tensor W \to U$, sodass $B\circ\tau = b$.
  \end{statements}
\end{proposition}

\begin{proposition}
  Seien $U,V,W$ Vektorräume und $b\in\Mult(V,W;U)$. Dann gilt
  \begin{statements}
  \item $V^*\tensor W \equiv \Hom(V,W)$. Insbesondere gilt $\dim
    V\tensor W = \dim V \cdot \dim W$.
  \item $V\tensor W \equiv W\tensor V$
  \item $V\tensor\K \equiv V$
  \item\label{tp:asso} $U\tensor(V\tensor W) \equiv (U\tensor V)\tensor W$
  \item $U\tensor(V\oplus W) \equiv U\tensor V \oplus U\tensor W$
  \end{statements}
\end{proposition}

\begin{bemerkung}
  Aus \ref{tp:asso} folgt die Wohldefiniertheit (bis auf Isomorphie)
  eines mehrfachen Tensorprodukts. Wir können festlegen, dass
  \begin{equation*}
    V_1\tensordots V_n := V_1\tensor (V_2\tensordots V_n).
  \end{equation*}
  Wir erhalten eine multilineare Abbildung
  \begin{equation*}
    \tilde\tau_n \colon V_1\dotted{\times}V_n \to
    V_1\tensordots V_n, (v^1,\dots,v^n) \mapsto \tau(v^1,\tilde\tau_{n-1}(v^2,\dots, v^n))
  \end{equation*}
  die  wir im folgenden auch einfach mit $\tau$ bezeichnen. Es gilt
  wie vorher auch, dass die Elemente $v^1\tensordots v^n :=
  \tau(v^1,\dots,v^n)$ ein Erzeugendensystem bilden. Die
  universelle Eigenschaft erweitert sich wie folgt:
\end{bemerkung}
\begin{proposition}[Universelle Eigenschaft des mehrfachen Tensorprodukts]
  Seien $V_1,\dots,V_n,W$ Vektorräume und $m\in
  \Mult(V_1,\dots,V_n;W)$. Dann gibt es eine eindeutig bestimmte
  lineare Abbildung $M\colon V_1\tensordots V_n \to W$, sodass $m =
  M\circ \tau$.
\end{proposition}

\begin{proposition}
  Seien $V_1,\dots,V_n,W$ Vektorräume. Dann gilt
  \begin{equation*}
    (V_1\tensordots V_n)^* \equiv V_1^*\tensordots V_n^*
  \end{equation*}
  und
  \begin{equation*}
    V_1^*\tensordots V_n^*\tensor W \equiv \Mult(V_1,\dots,V_n;W)
  \end{equation*}
\end{proposition}


\subsection{Tensoralgebra und äußere Algebra}
\label{sec:tauv}

\begin{definition}
  Wir definieren
  \begin{equation*}
    T^0(V) := \K,\, T^n(V) := V\tensor T^{n-1}(V)\text{ und }
    T^{(p,q)}(V) := T^p(V) \tensor T^q(V^*).
  \end{equation*}
  und identifizieren kanonisch $T^{(p,0)}(V)$ mit $T^p(V)$ und
  $T^{(0,p)}(V)$ mit $T^p(V^*)$.
  Es heißt
  \begin{equation*}
    T(V) := \bigoplus_{k\geq 0} T^k(V)
  \end{equation*}
  die \emph{Tensoralgebra von $V$} und
  \begin{equation*}
    \mathcal{T}(V) := \bigoplus_{p,q \geq 0} T^{(p,q)}(V)
  \end{equation*}
  die \emph{erweiterte Tensoralgebra von $V$}.

  Wir definieren $\tensor\colon \mathcal{T}(V)\times \mathcal{T}(V)
  \to \mathcal{T}(V)$ als die bilineare Abbildung, die eindeutig durch
  \begin{equation*}
    (v^1\tensordots v^k\tensor \nu_1\tensordots \nu_l,
    w^1\tensordots w^p\tensor \mu_1\tensordots \mu_q) \mapsto
    v^1\tensordots v^k\tensor w^1\tensordots w^p \tensor
    \nu_1\tensordots \nu_l \tensor \mu_1 \tensordots \mu_q
  \end{equation*}
  bestimmt ist.
\end{definition}

\begin{proposition}
  Die erweiterte Tensoralgebra ist eine $\Z$-graduierte kommutative $\K$-Algebra
  mit $1$. Die Tensoralgebra ist eine $\Z$-graduierte Unteralgebra der
  erweiterten Tensoralgebra.
\end{proposition}

\begin{definition}
  Bezeichne $S_k$ die Menge der Permutationen der Menge
  $\set{1,\dots,k}$. Eine Permutation $\sigma \in S_k$ induziert eine
  lineare Abbildung $\sigma \colon T^k(V) \to T^k(V)$, die durch
  \begin{equation*}
    v^1\tensordots v^k \mapsto v^{\sigma(1)}\tensordots v^{\sigma(k)}
  \end{equation*}
  eindeutig definiert ist.

  Wir definieren
  \begin{equation*}
    \Lambda^k(V) := \set{ t\in T^k(V) \mid \sigma(t) =
      \sgn(\sigma)\cdot t }
  \end{equation*}
  die Menge der \emph{alternierenden Tensoren vom Grad $k$} und die
  \emph{äußere Algebra von $V$} ist definiert als
  \begin{equation*}
    \Lambda(V) := \bigoplus_{k\geq 0}\Lambda^k(V)
  \end{equation*}
  

  Weiters bezeichne
  \begin{equation*}
    \Alt_k \colon T^k(V) \to T^k(V), t \mapsto
    \begin{cases}
      \frac{1}{k!}\sum_{\sigma\in S_k} \sgn(\sigma) \sigma(t) & 0\leq
      k \leq \dim V \\
      0 & \sonst
    \end{cases}
  \end{equation*}
  den (manchmal) sogenannten \emph{Alterator}.
\end{definition}

\begin{proposition}
  Sei $\sigma \in S_k$ eine Permutation. Es gilt
  \begin{statements}
  \item $\sigma(\Alt_k(t)) = \Alt_k(\sigma(t)) =
    \sgn(\sigma)\Alt_k(t)$.
  \item $\Alt_k \colon T^k(V) \to \Lambda^k(V)$ ist eine surjektive
    Projektion. Insbesondere sind also Elemente der Form
    \begin{equation*}
      v^1\dotted{\wedge} v^k := k!\Alt_k(v^1\tensordots v^k)
    \end{equation*}
    für $v^1,\dots,v^k\in V$ ein Erzeugendensystem für $\Lambda^k(V)$.
  \item $\dim \Lambda^k(V) = \binom{\dim V}{k}$
  \end{statements}
\end{proposition}

\begin{definition}
  Wir definieren das \emph{äußere Produkt} als die bilineare Abbildung
  $\Lambda(V\times \Lambda(V) \to \Lambda(V)$, die durch
  \begin{equation*}
    (v^1\wedgedots v^k) \wedge (w^1\wedgedots w^l) :=
    \frac{(k+l)!}{k!l!}\Alt_{k+l}((v^1\wedgedots v^k) \tensor
    (w^1\wedgedots w^l))
  \end{equation*}
  eindeutig bestimmt ist.
\end{definition}

\begin{proposition}
  Es gilt
  \begin{statements}
  \item Das äußere Produkt ist assoziativ.
  \item Das äußere Produkt ist ``superkommuativ'', das heißt für
    $\omega \in \Lambda^k(V)$ und $\eta \in \Lambda^l(V)$ gilt
    \begin{equation*}
      \omega \wedge \eta = (-1)^{kl}\eta\wedge \omega
    \end{equation*}
  \end{statements}
\end{proposition}

%%% Local Variables: 
%%% mode: latex
%%% TeX-master: "main"
%%% End: 



\end{document}

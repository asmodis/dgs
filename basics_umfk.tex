\section{Produkt- und Untermannigfaltigkeiten}
\label{sec:prumfk}

\begin{proposition}
  Seien $M_1^{m_1},\dots,M_n^{m_n}$ Mannigfaltigkeiten. Dann gibt es
  eine eindeutig bestimmte glatte Struktur auf $M :=
  M_1\times\dots\times M_n$, sodass die Projektionen $\pi_i \colon M
  \to M_i$ Submersionen sind und $\dim M = \sum_{i=1}^n m_i$.
\end{proposition}

\begin{definition}
  Seien $M_1^{m_1},\dots,M_n^{m_n}$ Mannigfaltigkeiten. Die obige
  eindeutig bestimmte glatte  Struktur heißt
  \emph{Produktmannigfaltigkeitsstruktur}. Wir vereinbaren, dass wir
  ein Produkt von Mannigfaltigkeiten mit dieser Struktur versehen,
  sofern nichts anderes erwähnt wird.
\end{definition}

\begin{proposition}
  Seien $M_1,\dots,M_n,P$ Mannigfaltigkeiten und $M :=
  M_1\times\dots\times M_n$. Dann ist $f\colon P \to M$ genau dann
  $C^k$, falls $\pi_i\circ f \colon P \to M_i$ für $i=1,\dots,n$ $C^k$
  ist, wobei $\pi_i$ die kanonische Projektion bezeichnet.
\end{proposition}

\begin{definition}
  Sei $M^m$ eine Mannigfaltigkeit und $N \subseteq M$. 
  Man nennt $N$ \emph{Untermannigfaltigkeit von $M$ der Kodimension
    $k$}, falls es um jeden Punkt $p\in N$ eine Karte $(U,x)$ gibt, sodass
  \begin{equation*}
    x(U\cap N) = \R^{m-k}\times\set{0}^{k} \cap x(U).
  \end{equation*}
  Eine solche Karte heißt \emph{angepasst an $N$}.
\end{definition}

\begin{proposition}
  Sei $M^m$ eine Mannigfaltigkeit und $N\subset M$ eine
  Untermannigfaltigkeit der Kodimension $k$. Dann gibt es eine
  eindeutige glatte Struktur auf $N$, sodass $N$ eine
  $m-k$-dimensionale Mannigfaltigkeit wird und die kanonische
  Inklusion $\incl\colon N \to M$ eine Einbettung ist.
\end{proposition}

\begin{korollar}
  Sei $M$ eine Mannigfaltigkeit und $O\subseteq M$ offen. Dann gibt es
  eine eindeutig bestimmte glatte Struktur auf $O$, sodass
  $\incl\colon O \to M$ eine Einbettung ist.

  Wir werden offene Teilmengen mit dieser Struktur versehen, wenn
  nichts anderes erwähnt wird.
\end{korollar}

\begin{satz}
  Sei $M^m$ eine Mannigfaltigkeit und $N\subseteq M$. Dann sind äquivalent
  \begin{statements}
  \item $N$ ist eine Untermannigfaltigkeit der Kodimension $k$.
  \item Zu jedem Punkt $p\in N$ gibt es eine in $M$ offene Umgebung
    $U\subset M$ von $p$ und einen Diffeomorphismus $\Phi\colon U \to
    \Phi(U) \subseteq \R^m$, sodass $\Phi(U\cap N) = \Phi(U)\cap
    \R^{m-k}\times\set{0}^k$.
  \item Zu jedem Punkt $p\in N$ gibt es eine in $M$ offene Umgebung
    $U\subset M$ von $p$ und eine Submersion $g\colon U \to \R^k$,
    sodass $U\cap N = g^{-1}(\set{0})$. Man nennt $g$ auch
    \emph{lokale Beschreibung von $N$ um $p$}.
  \item Zu jedem Punkt $p\in N$ gibt es eine in $M$ offene Umgebung
    $U\subset M$ von $p$, eine Umgebung $O\subset \R^{m-k}$ der $0$
    und eine Immersion $\phi\colon O \to M$,
    sodass $\phi(O) = U\cap N$ und $\phi(0) = p$. Man nennt $\phi$
    auch \emph{lokale Parametrisierung um $p$.}
  \end{statements}
\end{satz}

\begin{proposition}
  Seien $M^m,N^n$ Mannigfaltigkeiten und $f\colon M \to N$ glatt. Sei
  $q\in N$ und $P := f^{-1}(\set{q})$ nichtleer. Ist $f$ eine
  Submersion auf $P$,, so ist $P$ eine Untermannigfaltigkeit von $M$
  der Kodimension $n$.
\end{proposition}

\begin{proposition}
  Seien $M,P$ Mannigfaltigkeiten und $f\colon M\to P$ glatt.
  \begin{statements}
  \item Ist $N\subset M$ eine Untermannigfaltigkeit, dann ist auch
    $f|_N\colon N \to P$ glatt.
  \item Ist $N\subset P$ eine Untermannigfaltigkeit und ist
    $f(M)\subseteq N$, dann ist auch
    $f\colon M \to N$ glatt.
  \end{statements}
\end{proposition}

\begin{proposition}
  Sei $M$ eine Mannigfaltigkeit und $N\subseteq M$ eine
  Untermannigfaltigkeit. Dann ist für $p\in N$ die Abbildung $d_p\incl
  \colon T_pN \to T_pM$ injektiv. Man kann also kanonisch $T_pN$ als
  Untervektorraum von $T_pM$ auffassen.

  Ist $g$ eine lokale Beschreibung um $p$ und $\phi$ eine lokale
  Parametrisierung um $p$ mit $\phi(0) = p$, dann gilt
  \begin{equation*}
    T_pN = \ker d_p g = \Img d_0\phi
  \end{equation*}
\end{proposition}

\begin{definition}
  Seien $M,N$ Mannigfaltigkeiten und $f\colon M \to N$ glatt. Falls
  $f$ eine Immersion ist, so heißt $f\colon M\to N$ (beziehungsweise
  $f(M)$) \emph{immersierte Untermannigfaltigkeit}. Falls $f$ eine
  Einbettung ist, so heißt $f\colon M \to N$ (beziehungsweise $f(M)$)
  \emph{eingebettete Untermannigfaltigkeit}.
\end{definition}

\begin{proposition}
  Seien $M,N$ Mannigfaltigkeiten und $f\colon M\to N$ eine Einbettung. Dann ist
  $f(M)$ eine Untermannigfaltigkeit von $N$ mit Kodimension $\dim N -
  \dim M$. Ist umgekehrt $M\subset N$ eine Untermannigfaltigkeit, so
  ist $\incl\colon M \to N$ eine eingebettete Untermannigfaltigkeit.

  \textsc{Sprechweise:} Wir werden daher in Zukunft nicht mehr
  zwischen Untermannigfaltigkeiten und eingebettetn
  Untermannigfaltigkeiten unterscheiden.
\end{proposition}

%%% Local Variables: 
%%% mode: latex
%%% TeX-master: "main"
%%% End: 

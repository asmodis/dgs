\section{Mannigfaltigkeiten}
\label{sec:mfk}

\begin{definition}
  Sei $M$ eine nichtleere Menge und $m\in\N_0$. Ein Tupel $(U,\phi)$ heißt \emph{Karte auf
    $M$}, falls
  \begin{properties}
  \item $U \subset M$,
  \item $\phi\colon U \to \R^m$ eine injektive Abbildung ist und
  \item $\phi(U)$ eine offene Teilmenge des $\R^m$ ist.
  \end{properties}
  Man nennt dann $U$ \emph{Kartengebiet} und $\phi$
  \emph{Koordinatenabbildung} ($\phi^{-1}$ wird auch \emph{lokale
  Parametrisierung} genannt). Ist $p\in U \subset M$, so sagt man
auch, dass $(U,\phi)$ eine Karte \emph{um $p$} ist und falls $\phi(p)
= 0$, so sagt man, dass $(U,\phi)$ eine \emph{um $p$ zentrierte} Karte ist. 

  Eine Familie von Karten $A = \set{(U_i,\phi_i)}_{i\in  I}$ heißt
  \emph{(glatter) Atlas auf $M$}, falls
  \begin{properties}
  \item $\exists m \in \N_0\ \forall i\in I\colon \phi_i(U_i) \subset \R^m$,
  \item $\bigcup_{i\in I} U_i = M$,
  \item $\phi_i(U_i\cap U_j), \phi_j(U_i\cap U_j)$ sind offen und
  \item $\phi_i\circ \phi_j^{-1} \colon \phi_j(U_i\cap U_j) \to
    \phi_i(U_i\cap U_j)$ ist $C^\infty$.
  \end{properties}

  Ein Atlas $A$ heißt \emph{maximal}, falls für jede Karte $(U,\phi)$,
  für die $A\cup \set{(U,\phi)}$ ein glatter Atlas ist, bereits gilt, dass
  $(U,\phi) \in A$. Ein maximaler glatter Atlas heißt auch
  \emph{glatte Struktur}.

  Ein Tupel $(M,D)$ heißt \emph{(glatte) Mannigfaltigkeit}, falls $D$
  eine glatte Struktur ist. Eine Karte $(U,\phi)$ heißt \emph{zulässig},
  falls $(U,\phi)\in D$. Die \emph{Dimension} $\dim M$ einer
  Mannigfaltigkeit ist definiert als $\dim \phi(U)$ für eine zulässige
  Karte (nach Definition eines Atlanten ist die Dimension wohldefiniert). 
\end{definition}

\begin{proposition}\label{prop:basic:atlasbestimmtstruktur}
  Sei $A$ ein Atlas. Dann gibt es genau eine glatte Struktur $D$,
  sodass $A\subseteq D$.

  \textsc{Notation:} Für dieses $D$ schreiben wir auch $[A]$.
\end{proposition}

\begin{proposition}
  Sei $(M,D)$  eine Mannigfaltigkeit und $p\in M$. Dann gibt es für
  alle $0 < r \leq \infty$ eine zulässige um $p$ zentrierte Karte
  $(U,\phi)$ mit $\phi(U) = B_r(0)$.
\end{proposition}

\begin{definition}
  Sei $A$ ein Atlas. Die Initialtopologie bezüglich $A$ heißt
  \emph{Atlastopologie}, in Zeichen $\tau(A)$. Ist $A$ maximal so heißt die Atlastopologie
  auch \emph{Mannigfaltigkeitstopologie}.

  Eine Mannigfaltigkeit $(M,D)$ ist immer mit der
  Mannigfaltigkeitstopologie versehen.
\end{definition}

\begin{proposition}
  Seien $A,B$ Atlanten, sodass $A\cup B$ wieder ein Atlas ist. Dann
  gilt $\tau(A) = \tau(B)$. Insbesondere gilt also $\tau(A) =
  \tau([A])$, die Mannigfaltigkeitstopologie wird also durch eine
  geeignete Atlastopologie festgelegt.
\end{proposition}

\begin{proposition}
  Sei $(M,D)$ eine Mannigfaltigkeit und $\tau$ eine Topologie auf
  $M$. Dann gilt $\tau = \tau(D)$, falls für jede zulässige Karte
  $(U,\phi)$ gilt, dass
  \begin{properties}
  \item $U$ eine offene Menge bezüglich $\tau$ ist und
  \item $\phi \colon U \to \phi(U)$ ein Homeomorphismus bezüglich
    $\tau$ ist.
  \end{properties}
\end{proposition}

\begin{proposition}
  Sei $(M,D)$ eine Mannigfaltigkeit. Dann hat $M$ abzählbare
  Topologie, falls es einen abzählbaren Atlas $A$ mit $D=[A]$ gibt.

  $M$ ist hausdorffsch, falls es für alle Punkte $p,q\in M$ entweder
  eine zulässige Karte $(U,\phi)$ gibt mit $p,q\in U$ oder es zwei
  zulässige Karten $(U,\phi),(V,\psi)$ gibt mit $p\in U$, $q\in V$ und
  $U\cap V = \emptyset$.
\end{proposition}

\begin{proposition}
  Sei $(M,D)$ eine Mannigfaltigkeit. Hat $M$ abzählbare Topologie und
  ist hausdorffsch, so ist $M$ parakompakt.
\end{proposition}

\textbf{VEREINBARUNG:} Ab jetzt gilt immer, dass die
Mannigfaltigkeitstopologie abzählbar und hausdorffsch ist. Wenn von
Karten die Rede ist, sind immer zulässige Karten gemeint.

Wir werden in Zukunft die glatte Struktur nicht mehr unbedingt mit
angeben und erlauben uns, die Dimension einer Mannigfaltigkeit als
Hochzahl anzugeben, das heißt $M^m$ bezeichnet eine Mannigfaltigkeit
der Dimension $m$ (außer es ist aus dem Kontext klar, dass ein
$m$-faches kartesisches Produkt gemeint ist).

%%% Local Variables: 
%%% mode: latex
%%% TeX-master: "main"
%%% End: 


\section{Grundlagen}
\label{sec:basics}

\subsection{Mannigfaltigkeiten}
\label{sec:mfk}

\begin{definition}
  Sei $M$ eine nichtleere Menge und $m\in\N_0$. Ein Tupel $(U,\phi)$ heißt \emph{Karte auf
    $M$}, falls
  \begin{properties}
  \item $U \subset M$,
  \item $\phi\colon U \to \R^m$ eine injektive Abbildung ist und
  \item $\phi(U)$ eine offene Teilmenge des $\R^m$ ist.
  \end{properties}
  Man nennt dann $U$ \emph{Kartengebiet} und $\phi$
  \emph{Koordinatenabbildung} ($\phi^{-1}$ wird auch \emph{lokale
  Parametrisierung} genannt). Ist $p\in U \subset M$, so sagt man
auch, dass $(U,\phi)$ eine Karte \emph{um $p$} ist und falls $\phi(p)
= 0$, so sagt man, dass $(U,\phi)$ eine \emph{um $p$ zentrierte} Karte ist. 

  Eine Familie von Karten $A = \set{(U_i,\phi_i)}_{i\in  I}$ heißt
  \emph{(glatter) Atlas auf $M$}, falls
  \begin{properties}
  \item $\exists m \in \N_0\ \forall i\in I\colon \phi_i(U_i) \subset \R^m$,
  \item $\bigcup_{i\in I} U_i = M$,
  \item $\phi_i(U_i\cap U_j), \phi_j(U_i\cap U_j)$ sind offen und
  \item $\phi_i\circ \phi_j^{-1} \colon \phi_j(U_i\cap U_j) \to
    \phi_i(U_i\cap U_j)$ ist $C^\infty$.
  \end{properties}

  Ein Atlas $A$ heißt \emph{maximal}, falls für jede Karte $(U,\phi)$,
  für die $A\cup \set{(U,\phi)}$ ein glatter Atlas ist, bereits gilt, dass
  $(U,\phi) \in A$. Ein maximaler glatter Atlas heißt auch
  \emph{glatte Struktur}.

  Ein Tupel $(M,D)$ heißt \emph{(glatte) Mannigfaltigkeit}, falls $D$
  eine glatte Struktur ist. Eine Karte $(U,\phi)$ heißt \emph{zulässig},
  falls $(U,\phi)\in D$. Die \emph{Dimension} $\dim M$ einer
  Mannigfaltigkeit ist definiert als $\dim \phi(U)$ für eine zulässige
  Karte (nach Definition eines Atlanten ist die Dimension wohldefiniert). 
\end{definition}

\begin{proposition}\label{prop:basic:atlasbestimmtstruktur}
  Sei $A$ ein Atlas. Dann gibt es genau eine glatte Struktur $D$,
  sodass $A\subseteq D$.

  \textsc{Notation:} Für dieses $D$ schreiben wir auch $[A]$.
\end{proposition}

\begin{proposition}
  Sei $(M,D)$  eine Mannigfaltigkeit und $p\in M$. Dann gibt es für
  alle $0 < r \leq \infty$ eine zulässige um $p$ zentrierte Karte
  $(U,\phi)$ mit $\phi(U) = B_r(0)$.
\end{proposition}

\begin{definition}
  Sei $A$ ein Atlas. Die Initialtopologie bezüglich $A$ heißt
  \emph{Atlastopologie}, in Zeichen $\tau(A)$. Ist $A$ maximal so heißt die Atlastopologie
  auch \emph{Mannigfaltigkeitstopologie}.

  Eine Mannigfaltigkeit $(M,D)$ ist immer mit der
  Mannigfaltigkeitstopologie versehen.
\end{definition}

\begin{proposition}
  Seien $A,B$ Atlanten, sodass $A\cup B$ wieder ein Atlas ist. Dann
  gilt $\tau(A) = \tau(B)$. Insbesondere gilt also $\tau(A) =
  \tau([A])$, die Mannigfaltigkeitstopologie wird also durch eine
  geeignete Atlastopologie festgelegt.
\end{proposition}

\begin{proposition}
  Sei $(M,D)$ eine Mannigfaltigkeit und $\tau$ eine Topologie auf
  $M$. Dann gilt $\tau = \tau(D)$, falls für jede zulässige Karte
  $(U,\phi)$ gilt, dass
  \begin{properties}
  \item $U$ eine offene Menge bezüglich $\tau$ ist und
  \item $\phi \colon U \to \phi(U)$ ein Homeomorphismus bezüglich
    $\tau$ ist.
  \end{properties}
\end{proposition}

\begin{proposition}
  Sei $(M,D)$ eine Mannigfaltigkeit. Dann hat $M$ abzählbare
  Topologie, falls es einen abzählbaren Atlas $A$ mit $D=[A]$ gibt.

  $M$ ist hausdorffsch, falls es für alle Punkte $p,q\in M$ entweder
  eine zulässige Karte $(U,\phi)$ gibt mit $p,q\in U$ oder es zwei
  zulässige Karten $(U,\phi),(V,\psi)$ gibt mit $p\in U$, $q\in V$ und
  $U\cap V = \emptyset$.
\end{proposition}

\begin{proposition}
  Sei $(M,D)$ eine Mannigfaltigkeit. Hat $M$ abzählbare Topologie und
  ist hausdorffsch, so ist $M$ parakompakt.
\end{proposition}

\textbf{VEREINBARUNG:} Ab jetzt gilt immer, dass die
Mannigfaltigkeitstopologie abzählbar und hausdorffsch ist. Wenn von
Karten die Rede ist, sind immer zulässige Karten gemeint.

Wir werden in Zukunft die glatte Struktur nicht mehr unbedingt mit
angeben und erlauben uns, die Dimension einer Mannigfaltigkeit als
Hochzahl anzugeben, das heißt $M^m$ bezeichnet eine Mannigfaltigkeit
der Dimension $m$ (außer es ist aus dem Kontext klar, dass ein
$m$-faches kartesisches Produkt gemeint ist).

\subsection{Differenzierbare Abbildungen}
\label{sec:diffabb}

\begin{definition}
  Seien $M,N$ Mannigfaltigkeiten, $O\subset M$ offen und
  $p\in O$. Eine Abbildung $f\colon O \to N$
  heißt \emph{$k$-fach differenzierbar in $p$} (man sagt auch
  \emph{ist $C^k$ in $p$}), falls $f$ stetig ist und für alle Karten $(U,\phi)$ um $p$
  und alle Karten $(V,\psi)$ um $f(p)$ gilt, dass $\psi\circ
  f\circ\phi^{-1} \colon phi(U\cap f^{-1}(V)) \to \psi(f(U)\cap V)$
  $C^k$ in $\phi(p)$ ist. Die Abbildung $f$ ist
  \emph{$C^k$ auf $O$}, falls für alle $p\in O$ gilt, dass $f$ $C^k$
  in $p$ ist. Wir bezeichnen mit $C^k(O,N)$ die Abbildungen $f\colon O
  \to N$ die auf ganz $O$ $C^k$ sind und speziell sei $C^k(O) :=
  C^k(O,\R)$. 

  Eine Abbildung $f\colon O\to N$ heißt \emph{$C^k$-Diffeomorphismus},
  falls $f$ bijektiv ist, $f$ $C^k$ auf $O$ ist und $f^{-1}$ $C^k$ auf
  $N$ ist.

  Zwei Mannigfaltigkeiten heißen \emph{$C^k$-diffeomorph}, falls es
  einen $C^k$-Diffeomorphismus zwischen ihnen gibt.
\end{definition}

\textbf{VEREINBARUNG:} Differenzierbar oder glatt heißt bei uns immer
$C^\infty$. Ein Diffeomorphismus ist glatt (falls nicht anders erwähnt).

\begin{proposition}
  Seien $M,N$ Mannigfaltigkeiten, $p\in M$ und $f\colon M \to
  N$ stetig. Damit $f$ $C^k$ in $p$ ist genügt es, dass es eine Karte
  $(U,\phi)$ um $p$ und eine Karte $(V,\psi)$ um $f(p)$ gibt, sodass
  $\psi\circ f\circ \phi^{-1}$ $C^k$ in $\phi(p)$ ist. 

  Insbesondere
  ist also $f\colon M \to \R$ $C^k$ in $p$, falls es eine Karte
  $(U,\phi)$ um $p$ gibt, sodass $f\circ \phi^{-1}$ $C^k$ um $\phi(p)$ ist.
\end{proposition}

\begin{proposition}
  Sei $M$ eine Mannigfaltigkeit und $(U,\phi)$ eine Karte. Dann ist
  $\phi\colon U \to \phi(U)$ ein Diffeomorphismus.
\end{proposition}

\begin{proposition}
  Seien $M,N,P$ Mannigfaltigkeiten, $f\colon M \to N$ und $g\colon
  N\to P$ Abbilungen. Ist $f$ $C^k$ in $p$ und $g$ ist $C^k$ in
  $f(p)$, so ist $g\circ f$ $C^k$ in $p$.
\end{proposition}

\begin{definition}
 Sei $M$ eine Mannigfaltigkeit. Eine
  \emph{Zerlegung der Eins auf $M$} ist eine
  Familie $\set{\epsilon_i}_{i\in I}$ von Abbildungen, sodass
  \begin{properties}
  \item Die Menge $\set{\supp\epsilon_i}_{i\in I}$ ist lokal endlich,
  \item $\epsilon_i \colon M \to \R$ ist glatt,
  \item $0\leq \epsilon_i \leq 1$ und
  \item $\sum_{i\in I}\epsilon_i(p) = 1$ für alle $p\in M$. 
  \end{properties}

  Ist $\mathcal{U_j}_{j\in J}$ eine offene Überdeckung von $M$, so
  heißt eine Zerlegung der Eins $\set{\epsilon_i}_{i\in I}$ der
  Überdeckung $\mathcal{U_j}_{j\in J}$ \emph{untergeordnet}, falls es
  für alle $i\in I$ ein $j\in J$ gibt, sodass $\supp\epsilon_i \subset
  U_j$.
\end{definition}

\begin{proposition}
  Sei $M$ eine Mannigfaltigkeit und $\mathcal{U_j}_{j\in J}$ eine
  offene Überdeckung von $M$. Dann gibt es eine abzählbare Zerlegung
  der Eins $\set{\epsilon_i}_{i\in I}$, die $\mathcal{U_j}_{j\in J}$ untergeorddnet ist, sodass
  $\supp\epsilon_i$ kompakt ist.

  Verzichtet man auf die Kompaktheit des Trägers, so gibt es eine
  Zerlegung der Eins $\set{\epsilon_j}_{j\in J}$ mit $\supp\epsilon_j
  \subset U_j$ (gleicher Index!) wobei höchstens abzählbar viele
  $\epsilon_j$ nicht identisch verschwinden.
\end{proposition}

\begin{korollar}
  Sei $M$ eine Mannigfaltigkeit, $O\subset M$ offen und $A \subset O$
  abgeschlossen. Dann gibt es eine glatte Abbildung $h\colon M \to
  \R$, sodass $h|_A = 1$ und $h|_{M\setminus U} = 0$ und $0 \leq h
  \leq 1$.

  Eine solche Abbildung heißt \emph{Hutfunktion}.
\end{korollar}

\subsection{Der Tangentialraum an einen Punkt}
\label{sec:tpm}

Sei $M^m$ eine Mannigfaltigkeit und $p \in M$. Das Ziel dieses
Abschnitts ist es, einen Vektorraum zu definieren, der ``tangential''
an $M$ ist. Ist $M$ etwa als eine Teilmenge eines $\R^n$ angegeben
(etwa eine Sphäre), so hat man eine intuitive Vorstellung, welchen
(affine) Teilraum des $\R^n$ man als ``tangential in $p$'' bezeichnet.

Uns stellt sich dabei nun folgendes Problem: wie kann man den
Tangentialraum an eine \Mfk\ definieren, ohne auf einen umgebenden
Raum zurückzugreifen (man nennt dies auch ``intrinsisch'')? Wir wollen
vier verschiedene Ideen vorstellen und dann zeigen, dass diese vier
Vektorräume kanonisch isomorph sind, also im Prinzip denselben
Vektorraum beschreiben.

\textbf{Idee 1 (geometrisch):} Ist $\alpha \colon (-1,1) \to M$ eine
glatte Abbildung mit $\alpha(0) = p$ (``ein Weg durch $p$''), so
bestimmt diese Abbildung anschaulich einen Vektor, der Tangential an
$M$ liegt (den ``Geschwindigkeitsvektor'' der Kurve). Diesen kann man
``messen'', indem man eine Karte $(U,x)$ um $p$ wählt und dann
\begin{equation*}
  \eval\ddt x\circ \alpha(t)\at{t=0}
\end{equation*}
berechnet. Da es natürlich mehrere Kurven mit demselben
``Geschwindigkeitsvektor'' gibt, muss man diese noch miteinander
identifizieren. Präzisieren wir diese Idee:
\begin{definition}
  Eine glatte Abbildung $\alpha \colon (-1,1) \to M$ mit $\alpha(0) =
  p$ heißt \emph{glatter Weg durch $p$} oder auch \emph{glatte Kurve
    durch $p$}.

  Zwei glatte Kurven $\alpha,\beta$ durch $p$ berühren sich
  \emph{tangential} oder \emph{in erster Ordnung}, in Zeichen $\alpha
  \sim_1 \beta$, falls für alle Karten $(U,x)$ um $p$ gilt, dass
  \begin{equation}\label{eq:firstordercontact}
    \eval\ddt x\circ \alpha(t)\at{t=0} = \eval\ddt x\circ \beta(t)\at{t=0}
  \end{equation}
  und wir schreiben
  \begin{equation*}
    \dot\alpha(0) := \set{ \beta \mid \beta\text{ glatte Kurve durch
        $p$ und } \alpha \sim_1 \beta }
  \end{equation*}
\end{definition}

\begin{proposition}
  Es gilt
  \begin{statements}
  \item Die Relation $\sim_1$ ist eine Äquivalenzrelation.
  \item Gilt Gleichung \eqref{eq:firstordercontact} für eine Karte um
    $p$, so gilt sie bereits für alle Karten um $p$.
  \end{statements}
\end{proposition}
\begin{bemerkung}
  Man muss also die Gleichheit des ``Geschwindigkeitsvektors nur für
  eine Karte nachprüfen.
\end{bemerkung}

\begin{definition}
  Wir definieren nun
  \begin{equation*}
    T^{geo}_pM := \set{\dot\alpha(0) \mid \alpha \text{ glatte Kurve
        durch $p$}}
  \end{equation*}
  den \emph{(geometrischen) Tangentialraum an $p$}.
\end{definition}

Das Hauptproblem an diesem Ansatz ist es, dass man auf $T^{geo}_pM$ keine ganz
offensichtliche \VR-Struktur hat. Ein weiterer Abstraktionsschritt liefert

\textbf{Idee 2 (physikalisch):} Sei $\alpha$ eine Kurve durch
$p$ und $(U,x)$, $(V,y)$ zwei Karten um $p$. Ist nun
\begin{equation*}
  \xi := \eval\ddt x\circ \alpha(t)\at{t=0}\text{ und }\eta :=
  \eval\ddt y\circ \alpha(t)\at{t=0}
\end{equation*}
so rechhnet man leicht nach, dass $\xi = \Jac_{y(p)}(x\circ
y^{-1})\eta$ gilt. Die Idee ist es nun, Tupel von
Karten und Vektoren des $\R^m$ zu identifizieren, falls sich die
Vektoren mit der Jacobischen des Kartenwechsels transformieren
(physikalische Idee, da man Karten als ``Labore'' interpretieren kann
und Physiker Dinge darüber charakterisieren, wie sich
``Messergebnisse'' umrechnen). Wir erhalten
\begin{definition}
  Sei $A_p := \set{ (U,x,\xi) \mid (U,x) \text{ Karte um $p$}, \xi \in
    \R^m}$. Wir definieren die folgende Relation: $(U,x,\xi) \sim
  (V,y,\eta)$, falls $(U,x,\xi),(V,y,\eta) \in A_p$ und $\xi =
  \Jac_{y(p)}(x\circ y^{-1})\eta$ und bezeichnen
  \begin{equation*}
    [U,x,\xi] := \set{(V,y,\eta) \mid (U,x,\xi) \sim (V,y,\eta)}
  \end{equation*}
  und den \emph{(physikalischen) Tangentialraum an $p$} mit
  \begin{equation*}
    T^{phy}_pM := \set{ [U,x,\xi] \mid (U,x\xi) \in A_p }
  \end{equation*}
\end{definition}

\begin{proposition}
  Es gilt
  \begin{statements}
  \item Die Relation $\sim$ ist eine Äquivalenzrelation.
  \item Ist $(U,x)$ eine Karte um $p$, so gibt es für jedes Element
    $v\in T^{phy}_pM$ einen Represäntanten der Form $(U,x,\xi)$.
  \item Definiert man
    \begin{equation*}
      \lambda[U,x,\xi] + [V,y,\eta] := [U,x,\lambda\xi + \tilde\eta]
    \end{equation*}
    wobei $(U,x,\tilde\eta) \in [V,y,\eta]$, so erhält man eine
    wohldefinierte \VR-Struktur auf $T_p^{phy}M$.
  \item Für jede Karte $(U,x)$ um $p$ ist
    \begin{equation*}
      \set{[U,x,e^i] \mid i=1,\dots,\dim M}
    \end{equation*}
    eine Basis von $T^{phy}_pM$, wobei $e^1,\dots,e^m$ die
    Standardbasis des $\R^m$ bezeichnet. Insbesondere ist $\dim
    T^{phy}_pM = \dim M = m$.
  \end{statements}
\end{proposition}

Um den Zusammenhang zur ersten Idee herzustellen:
\begin{proposition}
  Sei $(U,x)$ eine Karte um $p$, $\lambda\in \R$ und $\dot\alpha(0),\dot\beta(0)\in
  T^{geo}_pM$. Definiere dann
  \begin{equation*}
    \gamma \colon (-1,1) \to M, t\mapsto x^{-1}\bigg( 
    \lambda\cdot \eval\ddt x\circ\alpha(t)\at{t=0}\cdot t + \eval\ddt
    x\circ\beta(t)\at{t=0} \cdot t + x(p)\bigg)
  \end{equation*}
  Dann ist $\gamma$ eine Kurve durch $p$ und durch
  \begin{equation*}
    \lambda \dot\alpha(0) + \dot\beta(0) := \dot\gamma(0)
  \end{equation*}
  wird eine wohldefinierte, von der Karte unabhängige \VR-Struktur auf
  $T^{geo}_pM$ definiert.

  Mit dieser \VR-Struktur gilt nun $T^{geo}_pM \equiv T^{phy}_pM$.
\end{proposition}



\subsection{Produkt- und Untermannigfaltigkeiten}
\label{sec:prumfk}



%%% Local Variables: 
%%% mode: latex
%%% TeX-master: "main"
%%% End: 

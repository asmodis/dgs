\section{Das Differential einer Abbildung in einem Punkt}
\label{sec:differential}

\begin{definition}
  Seien $M,N$ Mannigfaltigkeiten, $f\colon M \to N$ glatt in $p\in
  M$. Das \emph{(totale) Differential von $f$ in $p$} ist
  \begin{equation*}
    d_pf \colon T_pM \to T_{f(p)}N, \xi \mapsto d_pf(\xi)
  \end{equation*}
  wobei $d_pf(\xi)$ für $g\in C^\infty(N)$ wie folgt definiert ist
  \begin{equation*}
    d_pf(\xi)(g) := \xi(g\circ f)
  \end{equation*}
\end{definition}

\begin{proposition}
  Seien $M,N$ Mannigfaltigkeiten, $f\colon M \to N$ glatt in $p\in
  M$. Dann ist das Differential $d_pf$ eine wohldefinierte lineare Abbildung.
\end{proposition}

\begin{definition}
  Seien $M,N$ Mannigfaltigkeiten, $f\colon M \to N$ glatt in $p\in
  M$. Dann ist der \emph{Rang von $f$ in $p$}, $\rnk_p f$, wie folgt definiert
  \begin{equation*}
    \rnk_p f := \dim d_pf(T_pM),
  \end{equation*}
  also als Rang der linearen Abbildung $d_pf$. Man sagt, $f$ hat
  \emph{vollen Rang in $p$}, falls $\rnk_p f = \min(\dim M,\dim N)$.

  Hat die Abbildung $f$ vollen Rang, so heißt sie
  \begin{statements}
  \item \emph{Submersion in $p$}, falls  $\dim M \geq \dim N$,
  \item \emph{Immersion in $p$}, falls $\dim M \leq \dim N$ und
  \item \emph{Bimersion in $p$}, falls $\dim M = \dim N$.
  \end{statements}

  Eine glatte Abbildung $f\colon M \to N$ heißt \emph{Einbettung},
  falls $f$ eine injektive Immersion ist und $f\colon M \to f(M)$ ein
  Homeomorphismus ist wenn man $f(M)$ mit der Teilraumtopologie versieht.
\end{definition}

\begin{proposition}
  Seien $M,N$ Mannigfaltigkeiten, $f\colon M \to N$ glatt in $p\in
  M$. Dann ist $f$ genau dann eine
  \begin{statements}
  \item Submersion, falls $d_pf$ surjektiv ist.
  \item Immersion, falls $d_pf$ injektiv ist.
  \item Bimersion, falls $d_pf$ bijektiv ist.
  \end{statements}
\end{proposition}

\begin{proposition}
  Seien $M,N$ Mannigfaltigkeiten, $f\colon M \to N$ glatt. Dann ist
  $\rnk f$ unterhalbsstetig, das heißt ist $\rnk_p f = k$, so gibt es
  eine Umgebung $U$ von $p$, sodass $\rnk_q f \geq k$ für alle $q\in
  U$.

  Insbesondere haben Submersionen, Immersionen und Bimersionen einen
  lokal konstanten Rang.
\end{proposition}

\begin{satz}[Satz vom konstanten Rang]
  Seien $M^m,N^n$ Mannigfaltigkeiten, $f\colon M \to N$ glatt. Sei $p\in
  M$ und $f$ habe lokal um $p$ konstanten Rang, das heißt es gibt eine
  Umgebung $U$ um $p$, sodass $\rnk_qf = \rnk_pf$ für alle $q\in
  U$. Dann gibt es eine Karte $(U,x=(x^1,\dots,x^m))$ um $p$ und eine
  Karte $(V,y=(y^1,\dots,y^n))$ um $f(p)$, sodass $y\circ f = (x^1,\dots,x^k,0,\dots,0)$.
\end{satz}

\begin{korollar}[Normalformensatz]
  Seien $M^m,N^n$ Mannigfaltigkeiten, $f\colon M \to N$ glatt. Dann gibt
  es genau dann eine Karte $(U,x)$ um $p\in M$ und eine Karte $(V,y)$ um $f(p)$, sodass
  \begin{statements}
  \item $y\circ f\circ x^{-1} = \pr$, falls $f$ eine Submersion (und
    damit $m\geq n$) ist
    (wobei $\pr$ die Projektion $\R^m\to\R^n, (x^1,\dots,x^m) \mapsto
    (x^1,\dots,x^n)$ bezeichnet).
  \item $y\circ f\circ x^{-1} = \incl$, falls $f$ eine Immersion (und
    damit $m\leq n$) ist
    (wobei $\incl$ die Projektion $\R^m\to\R^n, (x^1,\dots,x^m) \mapsto
    (x^1,\dots,x^m,0,\dots,0)$ bezeichnet).
  \item $y\circ f\circ x^{-1} = \id$, falls $f$ eine Bimersion (und
    damit $m = n$) ist.
  \end{statements}
\end{korollar}

\begin{korollar}[Umkehrsatz]
  Seien $M,N$ Mannigfaltigkeiten, $f\colon M \to N$ glatt. Dann ist
  $f$ genau dann eine Bimersion in $p$, falls $f$ ein lokaler
  Diffeomorphismus um $p$ ist, das heißt es gibt eine Umgebung von
  $p$, sodass $f|_U\colon U \to f(U)$ ein Diffeomorphismus ist. 
\end{korollar}

\begin{proposition}
  Seien $M,N$ Mannigfaltigkeiten, $f\colon M \to N$ glatt mit vollem
  Rang. Ist $f$ injektiv, so ist $f$ eine Immersion.
\end{proposition}

\begin{proposition}
  Seien $M,N$ Mannigfaltigkeiten und $f\colon M \to N$ eine
  Submersion. Dann ist $f$ offen, das heißt für jede offene Menge
  $U\subseteq M$ gilt, dass $f(U) \subseteq N$ offen ist.
\end{proposition}

\begin{proposition}
  Seien $M,N,P$ Mannigfaltigkeiten und $f\colon M\to N$, $g\colon N
  \to P$ glatte Abbildungen. Dann ist $d_p g\circ f = d_{f(p)}g \circ d_pf$.
\end{proposition}

\begin{proposition}
  Seien $M,N$ Mannigfaltigkeiten, $M$ zusammenhängend und $f\colon M
  \to N$ glatt. Gilt für
  alle $p\in M$, dass $d_pf = 0$, so ist $f$ konstant. 
\end{proposition}

%%% Local Variables: 
%%% mode: latex
%%% TeX-master: "main"
%%% End: 

\section{Vektorbündel}
\label{sec:faserbl}

\begin{definition}
  Seien $E,M$ Mannigfaltigkeiten, $r\in\N$, $\K=\R,\C$ und $\pi\colon E \to M$ glatt. Man
  nennt $(E,M,\pi)$ \emph{$\K$-Vektorbündel vom Rang $r$ über $M$}, falls
  \begin{properties}
  \item $\pi\colon E \to M$ eine surjektive Submersion ist und
  \item für jeden Punkt $p\in M$ existiert eine Umgebung $U$ von $p$
    und ein Diffeomorphismus $\Phi = (\Phi_1,\Phi_2) \colon \pi^{-1}(U) \to U\times \K^r$,
    sodass $\Phi_1(e) = \pi(e)$ und $\Phi_p :=
    \Phi_2|_{\pi^{-1}(\set{p})} \colon \pi^{-1}(\set{p}) \to \K^r$
    ein linearer Isomorphismus ist.
  \end{properties}
  \textsc{Schreibweise:} Oft schreiben wir nur $\pi\colon
  E\to M$ oder sogar nur $E$ für ein Vektorbündel. Der Rang eines
  Bündels $E$ wird mit $\rnk E$ bezeichnet. Ist der Körper $\K$ aus
  dem Kontext klar (oder egal), so lassen wwir in weg.

  \textsc{Bezeichnungen:} Ist $\pi\colon E\to M$ ein Vektorbündel mit
  $\rnk E = r$, so heißt $E$ \emph{Totalraum}, $M$ \emph{Basisraum} und
  $\pi$ \emph{Bündelprojektion}. Ein Paar $(U,\Phi)$ wie in (ii)
  heißt \emph{Bündelkarte}, $U$ heißt \emph{lokal trivialisierende
    Umgebung} und $\Phi$ heißt \emph{lokale Trivialisierung}. Ein
  \emph{Bündelatlas} ist eine Familie von Bündelkarten, sodass die
  trivialisierenden Umgebungen ganz $M$ überdecken. Man sagt $E$
  \emph{trivialisiert über $U$}, falls es eine Bündelkarte $(U,\Phi)$ gibt.

  Ist $p$ in $M$ und $U\subset M$ offen, so bezeichnet $E_p :=
  \pi^{-1}(\set{p})$ die \emph{Faser über  $p$} und $E_U :=
  \pi^{-1}(U)$ das \emph{Teilbündel über $U$}.

  Ist $(U,\Phi)$ eine Bündelkarte, so bezeichnen $\pr_1,\pr_2\colon
  U\times \K^r$ die kanonischen Projektionen auf die erste und zweite Komponente.
\end{definition}

\begin{proposition}
  Sei $E\to M$ ein $\K$-Vektorbündel vom Rang $r$. Dann gilt $\dim E =
  \dim M + \dim_\R\K\cdot r$.
\end{proposition}

\begin{proposition}
  Sei $E\to M$ ein Vektorbündel und $U\subset M$ offen. Dann ist das
  Teilbündel $E_U$ ein Vektorbündel über $U$.
\end{proposition}

\begin{proposition}
  Sei $M$ eine Mannigfaltigkeit. Dann ist $\trivbl{\K^r} := M\times \K^r$ zusammen mit
  $\pi(p,f) = p$ ein $\K$-Vektorbündel vom Rang $r$. Man nennt
  $\trivbl{\K^r}$ das \emph{triviale $\K$-Vektorbündel}.
\end{proposition}

\begin{definition}
  Seien $\pi_E\colon E\to M$ und $\pi_F\colon F \to N$ zwei
  $\K$-Vektorbündel und $g\colon M\to N$ glatt. Eine Abbildung
  $G\colon E\to F$ heißt \emph{Bündelmorphismus entlang $g$}, falls
  $g\circ\pi_E = \pi_F\circ G$ und $G_p := G|_{E_p} \colon E_p \to
  F_{g(p)}$ linear ist.

  Ist $g$ ein Diffeomorphismus, so heißt eine bijektive Abbildung $G
  \colon E\to F$ \emph{Bündelisomorphismus
    entlang $g$}, falls $G$ ein Bündelmorphismus entlang $g$ und
  $G^{-1}$ ein Bündelmorphismus entlang $g^{-1}$ ist.

  Ist $M=N$ so heißt ein Bündelmorphismus entlang $\id$ einfach nur
  \emph{Bündel(homo)morphismus} und ein Bündelisomorphismus entlang
  $\id$ heißt \emph{Bündelisomorphismus}.

  Zwei Bündel heißen \emph{isomorph (entlang $g$)}, falls es einen
  Bündelisommorphismus (entlang $g$) gibt. Ein Bündel heißt
  \emph{trivialisierbar}, falls es isomorph zum trivialen Bündel ist.
\end{definition}

\begin{proposition}
  Ein Bündel $E\to M$ ist genau dann trivialisierbar, wenn es eine
  globale Bündelkarte $(M,\Phi)$ gibt.
\end{proposition}

\begin{definition}
  Sei $\pi\colon E\to M$ ein $\K$-Vektorbündel vom Rang $r$ und $(U_1,\Phi_1),
  (U_2,\Phi_2)$ zwei Bündelkarten um $p\in M$. Dann heißt
  \begin{equation*}
    g_{12} \colon U_1\cap U_2 \to \GL(\K^r), p\mapsto \pr_2(\Phi_2\circ \Phi_1^{-1}(p,\cdot))
  \end{equation*}
  der \emph{Bündelübergang von $(U_1,\Phi_1)$ nach $(U_2,\Phi_2)$}.
\end{definition}

\subsection*{Charakterisierung durch Kozykel}

\begin{definition}
  Sei $M$ eine Mannigfaltigkeit und $X$ eine Menge von Abbildungen,
  die abgeschlossen unter Komposition ist.
  Sei weiterhin $\set{U_i}_{i\in I}$ eine offene
  Überdeckung von $M$ und $z = \set{ g_{ij} \colon U_i\cap U_j \to X}$
  eine Familie von Abbildungen. Man nennt $z$ einen
  \emph{$1$-Kozykel mit Werten in $X$}, falls er die \emph{Kozykelbedingung}
  \begin{statements}
  \item $g_{ii} = \id$
  \item $g_{ij} \circ g_{jk} = g_{ik}$ auf $U_i\cap U_j\cap U_k$
  \end{statements}
  erfüllt.

  Eine Familie von Abbildungen $\set{h_i\colon U_i \to X}$ heißt
  \emph{$1$-Korand mit Werten in $X$}, falls die $h_i$ invertierbar
  sind. Zwei Kozykel $\set{g_{ij}}$ und
  $\set{\tilde{g}_{ij}}$ heißen \emph{kohomolog}, falls es einen
  Korand $\set{h_i}$ gibt, sodass $\tilde{g}_{ij} = h_i \circ g_{ij}
  \circ h_j^{-1}$ 
\end{definition}

\begin{proposition}
  Sei $\set{(U_i,\Phi_i)}_{i\in I}$ ein Bündelatlas eines $\K$-Vektorbündels
  $\pi\colon E \to M$ vom Rang $r$. Dann bilden die Bündelübergänge $g_{ij}$ einen
  $1$-Kozykel mit Werten in $\GL(\K^r)$. Man nennt diesen Kozykel auch
  \emph{Übergangskozykel (zur Überdeckung $\set{U_i}$)}.
\end{proposition}

\begin{proposition}
  Sei $M$ eine Mannigfaltigkeit und $\set{U_i}_{i\in I}$ eine offene
  Überdeckung von $M$. Ist $z = \set{g_{ij}}$ ein 1-Kozykel mit Werten
  in $\GL(\K^r)$, so gibt es ein $\K$-Vektorbündel vom Rang $r$,
  welches über $\set{U_i}$ trivialisiert und dort $z$ als
  Übergangskozykel hat.
\end{proposition}

\begin{proposition}
  Seien $E\to M$, $F\to M$ Vektorbündel die beide über der offenen
  Überdeckung $\set{U_i}_{i\in I}$ von $M$ trivialisieren. Dann sind $E$ und $F$ genau
  dann isomorph, wenn die jeweiligen Übergangskozykel kohomolog sind.
\end{proposition}

\begin{satz}
  Sei $M$ eine Mannigfaltigkeit und $\set{U_i}_{i\in I}$ eine offene
  Überdeckung von $M$. Ist $z = \set{g_{ij}}$ ein 1-Kozykel mit Werten
  in $\GL(\K^r)$, so gibt es ein bis auf Bündelisomorphie eindeutiges $\K$-Vektorbündel vom Rang $r$,
  welches über $\set{U_i}$ trivialisiert und dessen Übergangskozykel
  zu $z$ kohomolog ist.
\end{satz}

\subsection*{Rahmen und Schnitte}

\begin{definition}
  Sei $\pi \colon E\to M$ ein Vektorbündel vom Rang $r$. Eine
  Abbildung $s\colon M \to E$ heißt \emph{rauher Schnitt in/von $E$}, falls
  $\pi\circ s 0 \id$. Ein \emph{(glatter) Schnitt in/von $E$} ist ein
  rauher Schnitt, der glatt ist. Die Menge aller glatten Schnitte wird
  mit $\Gamma(E)$ bezeichnet. Man schreibt oft $s_p := s(p)$ für einen
  Schnitt $s\in \Gamma(E)$. Ein \emph{lokaler Schnitt über $U\subset
    M$} ist ein (globaler) Schnitt über $E_U$.

  Man nennt Schnitte $s_1,\dots,s_n$ \emph{linear unabhängig}, falls
  sie dies Punktweise sind. Ein \emph{(lokaler) Rahmen  über $U\subset
    M$} $\vbframe{\Psi}
  = (\Psi_1,\dots,\Psi_r)$ ist ein $r$-Tupel aus lokalen Schnitten
  $\Psi_i \in \Gamma(E_U)$, sodass
  $\vbframe{\Psi}_p = (\Psi_1(p),\dots,\Psi_r(p))$ für jedes $p\in U$
  eine Basis von $E_p$ ist.
\end{definition}

\begin{proposition}
  Für$s,t\in\Gamma(E)$ und $f\inC^\infty(M)$ wird durch $(fs + t)(p)
  := f(p)s(p) + t(p)$ die Menge $\Gamma(E)$ zu einem lokal freien $C^\infty(M)$-Modul.
\end{proposition}

\begin{proposition}
  Ist $E\to M$ ein Vektorbündel, so gibt es eine bijektive
  Korrespondenz zwischen Bündelkarten $(U,\Phi)$ und lokalen Rahmen
  $\vbframe{\Psi}$ über $U$ in dem Sinne, dass es zu jeder Bündelkarte
  einen lokalen Rahmen gibt und umgekehrt.
\end{proposition}

\subsection*{Konstruktion neuer Vektorbündel}



%%% Local Variables: 
%%% mode: latex
%%% TeX-master: "main"
%%% End: 

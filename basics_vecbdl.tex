\section{Vektorbündel}
\label{sec:faserbl}


\subsection{Grundlegende Begriffsbildungen}

\begin{definition}
  Seien $E,M$ Mannigfaltigkeiten. Ein Tripel $(\pi,E,M)$ heißt
  \emph{glattes Bündel über $M$}, falls $\pi \colon E\to M$ eine
  surjektive Submersion ist. Man nennt $E$ den \emph{Totalraum}, $M$
  den \emph{Basisraum} und bezeichnet $E_p := \pi^{-1}(\set{p})$ als
  die \emph{Faser von $E$ über $p$} ($p$ heißt auch \emph{Fußpunkt von
    $E_p$)} und für $U\subset M$ offen bezeichnet man $E_U :=
  \pi^{-1}(U)$ als das \emph{eingeschränkte Bündel über $U$}.

  Ein \emph{Schnitt über $U$} ist eine glatte Abbildung $s\colon U \to
  E$, sodass $\pi|_U\circ s = \id_U$. Ein \emph{(globaler) Schnitt}
  ist ein Schnitt über $M$. Wir definieren
  \begin{equation*}
    \Gamma(E_U) := \set{ s\colon U \to E \mid s\text{ Schnitt über } U }
  \end{equation*}
  und $\Gamma(E) := \Gamma(E_M)$.

  \textsc{Notation:} Für ein Bündel $(\pi,E,M)$ schreiben wir oft auch
  $\pi\colon E\to M$, $E\to M$ oder einfach nur $E$.
\end{definition}

\begin{definition}
  Sei $F$ eine Mannigfaltigkeit, $\pi\colon E \to M$ ein Bündel
  $U\subset M$ offen. Bezeichne $\pr_1\colon U\times F \to U, (u,f)
  \mapsto u$ und $\pr_2\colon U\times F \to U, (u,f)
  \mapsto f$.
  Man sagt \emph{$E$ trivialisiert über $U$ mit Fasertyp $F$}, falls es
  einen Diffeomorphismus $\Phi \colon E_U \to U\times F$ gibt, sodass $\pi
  = \pr_1\circ \Phi$. Man nennt dann $(U,\Phi)$ eine
  \emph{Bündelkarte} und $\Phi$ eine \emph{lokale
    Trivialisierung}. Man nennt $\Phi^v := \pr_2\circ \Phi \colon E_U
  \to F$ den \emph{Vektoranteil} der Trivialisierung.

  Man sagt $E$ ist \emph{lokal trivial mit Fasertyp $F$}, falls es
  eine offene Überdeckung $\set{U_i}_{i\in I}$ von $M$ gibt, sodass
  $E$ für alle $i\in I$ über $U_i$ trivialisiert. Ist $\Phi_i \colon
  E_{U_i}\to U_i\times F$ eine lokale Trivialisierung, so heißt
  $\set{(U_i,\Phi_i)}$ \emph{Bündelatlas}. 
\end{definition}

\begin{proposition}
  Sei $\pi \colon E \to M$ ein Bündel und $(U,\Phi)$ eine Bündelkarte
  um $p$. Dann ist die Abbildung $\Phi_p \colon E_p \to F, \xi \mapsto
  \Phi^v|_{E_p}(\xi)$ ein Diffeomorphismus.

  Insbesondere ist also für zwei Bündelkarten  $(U,\Phi),(\tilde
  U,\tilde \Phi)$ um $p$ die Abbildung
  $\tau_p := \Phi_p\circ\tilde{\Phi_p}^{-1}$ ein Diffeomorphismus. 
\end{proposition}

\begin{definition}
  Sei $\pi \colon E \to M$ ein lokal triviales Bündel mit Fasertyp $F$ und $(U,\Phi),(\tilde
  U,\tilde \Phi)$ zwei Bündelkarten um $p$. Die Abbildung $\tau \colon
  U \cap \tilde U \to \Diff(F), p \mapsto
  \Phi_p\circ\tilde{\Phi_p}^{-1}$ heißt \emph{Übergangsfunktion von
    $(U,\Phi)$ zu $(\tilde U,\tilde\Phi)$}. Ist
  $\set{(U_i,\Phi_i)}_{i\in I}$ ein Bündelatlas, so bezeichnet
  $g_{ij}$ die Übergangsfunktion von $(U_i,\Phi_i)$ zu $(U_j,\Phi_j)$.
\end{definition}

\begin{definition}
  Sei $\K = \R$ oder $\K = \C$. Ein Bündel $\pi\colon E\to M$ heißt
  \emph{$\K$-Vektorbündel vom Rang $r$}, falls $E$ lokal trivial mit
  Fasertyp $\K^r$ ist und für jeden Bündelatlas $\set{(U_i,\Phi_i)}$
  und jedes $p\in M$ gilt, dass für die Übergangsfunktion gilt, dass
  $g_{ij}(p)\in\GL(\K^r)$. 

  Man sagt auch \emph{reelles Vektorbündel} für $\K=\R$ und
  \emph{komplexes Vektorbündel} für $\K=\C$. Ein Vektorbündel vom Rang
  $1$ heißt auch \emph{Linienbündel}.
\end{definition}

\begin{proposition}
  Ist $E\to M$ ein $\K$-Vektorbündel vom Rang $r$, so trägt jede Faser $E_p$ eine
  natürliche Vektorraumstruktur, sodass $\Phi_p\colon E_p \to \K^r$
  linear wird. Es ist also jede Faser $E_p$ ein $r$-dimensionaler $\K$-Vektorraum.
\end{proposition}

\begin{proposition}
  Sei $E\to M$ ein lokal triviales Bündel mit Fasertyp $\K^r$. Dann
  ist $E$ genau dann ein $\K$-Vektorbündel vom Rang $r$, falls jede
  Faser $E_p$ die Struktur eines $r$-dimensionalen $\K$-Vektorraums
  hat und für jede Bündelkarte $(U,\Phi)$ die Abbildung $\Phi_p$
  linear ist.
\end{proposition}

\begin{proposition}
  Ist $E\to M$ ein $\K$-Vektorbündel vom Rang $r$, so ist $\dim E =
  \dim M + \dim_\R\K\cdot r$.
\end{proposition}

\begin{proposition}
  Sei $E\to M$ ein Vektorbündel und $U\subset M$ offen. Dann ist das
  Teilbündel $E_U$ ein Vektorbündel über $U$.
\end{proposition}

\begin{proposition}
  Sei $M$ eine Mannigfaltigkeit. Dann ist $\trivbl{\K^r} := M\times \K^r$ zusammen mit
  $\pi(p,f) = p$ ein $\K$-Vektorbündel vom Rang $r$. Man nennt
  $\trivbl{\K^r}$ das \emph{triviale $\K$-Vektorbündel}.
\end{proposition}

\begin{definition}
  Seien $\pi_E\colon E\to M$ und $\pi_F\colon F \to N$ zwei
  $\K$-Vek torbündel und $g\colon M\to N$ glatt. Eine Abbildung
  $G\colon E\to F$ heißt \emph{Bündelmorphismus entlang $g$}, falls
  $g\circ\pi_E = \pi_F\circ G$ und $G_p := G|_{E_p} \colon E_p \to
  F_{g(p)}$ linear ist.

  Ist $g$ ein Diffeomorphismus, so heißt eine bijektive Abbildung $G
  \colon E\to F$ \emph{Bündelisomorphismus
    entlang $g$}, falls $G$ ein Bündelmorphismus entlang $g$ und
  $G^{-1}$ ein Bündelmorphismus entlang $g^{-1}$ ist.

  Ist $M=N$ so heißt ein Bündelmorphismus entlang $\id$ einfach nur
  \emph{Bündel(homo)morphismus} und ein Bündelisomorphismus entlang
  $\id$ heißt \emph{Bündelisomorphismus}.

  Zwei Bündel heißen \emph{isomorph (entlang $g$)}, falls es einen
  Bündelisommorphismus (entlang $g$) gibt. Ein Bündel heißt
  \emph{trivialisierbar}, falls es isomorph zum trivialen Bündel ist.
\end{definition}

\begin{proposition}
  Ein Bündel $E\to M$ ist genau dann trivialisierbar, wenn es eine
  globale Bündelkarte $(M,\Phi)$ gibt.
\end{proposition}

\subsection*{Rahmen und Schnitte}

\begin{definition}
  Sei $E\to M$ ein Vektorbündel vom Rang $r$. Die Schnitte $s_1,\dots,
  s_k$ über $U$ heißen \emph{linear unabhängig über $U$}, falls die
  Vektoren $s_1(p),\dots,s_k(p)$ dies sind. Man nennt $\vbfram{\Psi} =
  (\Psi_1,\dots,\Psi_r)$ \emph{(lokalen) Rahmen über $U$}, falls die
  $\Psi_1,\dots, \Psi_r$ linear unabhängige Schnitte über $U$ sind,
  also punktweise eine Basis bilden. Ein \emph{globaler Rahmen} ist
  ein lokaler Rahmen über $M$. 
\end{definition}

\begin{proposition}
  Sei $E\to M$ ein $\K$-Vektorbündel vom Rang $r$ und $(U,\Phi)$ eine Bündelkarte und bezeichne $e_1,\dots, e_r$ eine
  Basis des $\K^r$. Definiert man $Psi_i(p) := \Phi^{-1}(p,e_i)$, so
  ist $\vbframe{\Psi} = (\Psi_1,\dots,\Psi_r)$ ein Rahmen über
  $U$. Man sagt dann, dass $\vbframe{\Psi}$ zu $(U,\Phi)$
  \emph{assoziiert} ist.
\end{proposition}

\begin{proposition}
  Sei $E\to M$ ein Vektorbündel und $\vbframe{\Psi}$ ein Rahmen über
  $U$. Dann gibt es eine Bündelkarte $(U,\Phi)$, sodass $\vbfram{Psi}$
  zu $(U,\Phi)$ assoziert ist.
\end{proposition}

\begin{proposition}
  Ein Vektorbündel $E\to M$ ist genau dann trivial, wenn es einen
  globalen Rahmen gibt.
\end{proposition}

\begin{proposition}
  Sei $E\to M$ ein Vektorbündel. Ein Schnitt $s\in \Gamma(E)$ ist
  genau dann glatt, wenn es zu jedem Rahmen $\vbframe{\Psi} =
  (\Psi_1,\dots,\Psi_r)$ über einer Menge $U$ glatte Abbildungen $f_1,\dots, f_n\in
  C^\infty(U)$ gibt, mit $s|_U = \sum_{i} f_i\Psi_i$.
\end{proposition}

\begin{proposition}
  Für$s,t\in\Gamma(E)$ und $f\inC^\infty(M)$ wird durch $(fs + t)(p)
  := f(p)s(p) + t(p)$ die Menge $\Gamma(E)$ zu einem lokal freien $C^\infty(M)$-Modul.
\end{proposition}

\begin{proposition}[TODO]
  Sei $E\to M$ ein Vektorbündel, $\vbframe{\Psi}$ ein Rahmen über $U$
  und $\vbframe{\Phi}$ ein Rahmen über $U'$. Ist $s\in\Gamma(E_{U\cap
    U'})$ mit $s = \sum_i f_i \Psi_i = \sum_i \tilde{f}_i \Phi_i$, so gilt
  \begin{equation*}
    f_i(p) = \tau(p)\tilde{f}_i(p)
  \end{equation*}
  wobei $\tau$ die Bündelübergangsfunktion der zu den Rahmen gehörigen
  lokalen Trivialisierungen bezeichnet.
\end{proposition}


\subsection*{Charakterisierung durch Kozykel}

\begin{definition}
  Sei $M$ eine Mannigfaltigkeit und $X$ eine Menge von Abbildungen,
  die abgeschlossen unter Komposition ist.
  Sei weiterhin $\set{U_i}_{i\in I}$ eine offene
  Überdeckung von $M$ und $z = \set{ g_{ij} \colon U_i\cap U_j \to X}$
  eine Familie von Abbildungen. Man nennt $z$ einen
  \emph{$1$-Kozykel mit Werten in $X$}, falls er die \emph{Kozykelbedingung}
  \begin{statements}
  \item $g_{ii} = \id$
  \item $g_{ij} \circ g_{jk} = g_{ik}$ auf $U_i\cap U_j\cap U_k$
  \end{statements}
  erfüllt.

  Eine Familie von Abbildungen $\set{h_i\colon U_i \to X}$ heißt
  \emph{$1$-Korand mit Werten in $X$}, falls die $h_i$ invertierbar
  sind. Zwei Kozykel $\set{g_{ij}}$ und
  $\set{\tilde{g}_{ij}}$ heißen \emph{kohomolog}, falls es einen
  Korand $\set{h_i}$ gibt, sodass $\tilde{g}_{ij} = h_i \circ g_{ij}
  \circ h_j^{-1}$ 
\end{definition}

\begin{proposition}
  Sei $\set{(U_i,\Phi_i)}_{i\in I}$ ein Bündelatlas eines $\K$-Vektorbündels
  $\pi\colon E \to M$ vom Rang $r$. Dann bilden die Bündelübergänge $g_{ij}$ einen
  $1$-Kozykel mit Werten in $\GL(\K^r)$. Man nennt diesen Kozykel auch
  \emph{Übergangskozykel (zur Überdeckung $\set{U_i}$)}.
\end{proposition}

\begin{proposition}
  Sei $M$ eine Mannigfaltigkeit und $\set{U_i}_{i\in I}$ eine offene
  Überdeckung von $M$. Ist $z = \set{g_{ij}}$ ein 1-Kozykel mit Werten
  in $\GL(\K^r)$, so gibt es ein $\K$-Vektorbündel vom Rang $r$,
  welches über $\set{U_i}$ trivialisiert und dort $z$ als
  Übergangskozykel hat.
\end{proposition}

\begin{proposition}
  Seien $E\to M$, $F\to M$ Vektorbündel die beide über der offenen
  Überdeckung $\set{U_i}_{i\in I}$ von $M$ trivialisieren. Dann sind $E$ und $F$ genau
  dann isomorph, wenn die jeweiligen Übergangskozykel kohomolog sind.
\end{proposition}

\begin{satz}
  Sei $M$ eine Mannigfaltigkeit und $\set{U_i}_{i\in I}$ eine offene
  Überdeckung von $M$. Ist $z = \set{g_{ij}}$ ein 1-Kozykel mit Werten
  in $\GL(\K^r)$, so gibt es ein bis auf Bündelisomorphie eindeutiges $\K$-Vektorbündel vom Rang $r$,
  welches über $\set{U_i}$ trivialisiert und dessen Übergangskozykel
  zu $z$ kohomolog ist.
\end{satz}


\subsection*{Konstruktion neuer Vektorbündel}

\begin{definition}
  Sei $E\to M$ ein Vektorbündel vom Rang $r$. Ein $F\subset M$ heißt
  \emph{Untervektorbündel von $E$ vom Rang $s$}, falls für alle $p\in
  M$ gilt, dass $F_p := E_p \cap F$ ein $s$-dimensionaler
  Vektorunterraum von $E_p$ ist und es für jedes $p\in M$ eine
  (sogenannte \emph{angepasste}) Bündelkarten $(U,\Phi)$ um $p$ gibt,
  sodass $\Phi(U\cap F) = U\times (\R^s\times\set{0}^{r-s})$
\end{definition}

\begin{proposition}
  Sei $E\to M$ ein Vektorbündel. Ein $F\subset E$ ist genau dann ein
  Untervektorbündel vom Rang $s$, falls jedes $p\in M$ eine offene
  Umgebung $U$ besitzt, sodass es $s$ linear unabhängige
  Schnitte $\Psi_1,\dots,\Psi_s$ über $U$ gibt, sodass $F_q = F\cap E_q$ die
  lineare Hülle von $\Psi_1(q),\dots,\Psi_s(q)$ für alle $q\in U$ ist.
\end{proposition}

\begin{definition}
  Sei $E\to M$ ein Vektorbündel vom Rang $r$ mit Bündelatlas
  $\set{(U_i,\Phi_i)}_{i\in I}$ und Übergangskozykel
  $\set{g_{ij}}$. Das \emph{duale Bündel $E^*\to M$} ist das durch den
  Kozykel $\set{g_{ij}^*}$ (bis auf Isomorphie) bestimmte
  Vektorbündel, wobei
  \begin{equation*}
    g_{ij}^*(p) := ((g_{ij}(p))^t)^{-1}
  \end{equation*}
\end{definition}

\begin{definition}
  Sei $E,\tilde E\to M$ Vektorbündel vom Rang $r$ und $\tilde r$ mit Bündelatlas
  $\set{(U_i,\Phi_i)}_{i\in I}$, $\set{(U_i,\tilde\Phi_i)}_{i\in I}$ und Übergangskozykel
  $\set{g_{ij}}$, $\set{\tilde g_{ij}}$. Das \emph{Summenbündel
    $E\oplus \tilde E\to M$} ist das durch den
  Kozykel $\set{g_{ij}^\oplus\colon U_i\cap U_j \to \GL(\K^{r+\tilde r}}$ (bis auf Isomorphie) bestimmte
  Vektorbündel, wobei
  \begin{equation*}
    g_{ij}^\oplus(p)(x_1,\dots,x_r,y_1,\dots,y_{\tilde r} :=
    (g_{ij}(p)(x_1,\dots,x_r},\tilde g_{ij}(y_1,\dots,y_{\tilde r}))
  \end{equation*}
\end{definition}

Tensor

%%% Local Variables: 
%%% mode: latex
%%% TeX-master: "main"
%%% End: 

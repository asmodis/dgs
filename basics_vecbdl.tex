\section{Faserbündel}
\label{sec:faserbl}

\begin{definition}
  Seien $E,M,F$ Mannigfaltigkeiten und $\pi\colon E \to M$ glatt. Man
  nennt $(E,M,\pi,F)$ \emph{Faserbündel über $M$ mit Fasertyp $F$} (oder auch \emph{$F$-Faserbündel}), falls
  \begin{properties}
  \item $\pi\colon E \to M$ eine surjektive Submersion ist und
  \item für jeden Punkt $p\in M$ existiert eine Umgebung $U$ von $p$
    und ein Diffeomorphismus $\Phi = (\Phi_1,\Phi_2) \colon \pi^{-1}(U) \to U\times F$,
    sodass $\Phi_1(e) = \pi(e)$ gilt.
  \end{properties}
  Dabei heißt $E$ der \emph{Totalraum}, $M$ der \emph{Basisraum} und
  $\pi$ die \emph{Bündelprojektion}. Ein Paar $(U,\Phi)$ wie in (ii)
  heißt \emph{Bündelkarte}, $U$ heißt \emph{lokal trivialisierende
    Umgebung} und $\Phi$ heißt \emph{lokale Trivialisierung}. Ein
  \emph{Bündelatlas} ist eine Familie von Bündelkarten, sodass die
  trivialisierenden Umgebungen ganz $M$ überdecken. Man sagt $E$
  \emph{trivialisiert über $U$}, falls es eine Bündelkarte $(U,\Phi)$ gibt.

  Ist $p$ in $M$ und $U\subset M$ offen, so bezeichnet $E_p :=
  \pi^{-1}(\set{p})$ die \emph{Faser über  $p$} und $E_U :=
  \pi^{-1}(U)$ das \emph{Teilbündel über $U$}.

  \textsc{Notation:} Wir schreiben statt des $4$-Tupels oft einfach
  nur $\pi\colon E\to M$ oder nur $E$ für ein Faserbündel.
\end{definition}

\begin{proposition}
  Seien $M,F$ Mannigfaltigkeiten. Dann ist $\trivbl{F} := M\times F$ zusammen mit
  $\pi(p,f) = p$ ein Faserbündel mit Fasertyp $F$.
\end{proposition}

\begin{proposition}
  Sei $\pi\colon E\to M$ ein $F$-Faserbündel, $p\in M$ und $(U,\Phi)$
  eine Bündelkarte um $p$. Dann ist $\Phi_p \colon E_p \to F, e
  \mapsto \pr_2(\Phi(e))$ ein Diffeomorphismus, wobei $\pr_2 \colon
  \set{p} \times F \to F$ die kanonische Projektion ist.
\end{proposition}

\begin{definition}
  Seien $M,F$ Mannigfaltigkeiten und $\pi\colon E\to M$ ein
  Faserbündel mit Fasertyp $F$. Das Bündel $\trivbl{F} := M\times F$
  heißt \emph{triviales Bündel vom Typ $F$} und man nennt $E$
  \emph{trivialisierbar}, falls es eine globale Bündelkarte gibt, das
  heißt einen Diffeomorphismus $\Phi \colon E \to U \times F$.
\end{definition}

\begin{definition}
  Sei $\pi\colon E\to M$ ein $F$-Faserbündel und $(U_1,\Phi_1),
  (U_2,\Phi_2)$ zwei Bündelkarten um $p\in M$. Dann heißt
  \begin{equation*}
    g_{12} \colon U_1\cap U_2 \to \Diff(F), p\mapsto \pr_2(\Phi_2\circ \Phi_1^{-1}(p,\cdot))
  \end{equation*}
  der \emph{Bündelübergang von $(U_1,\Phi_1)$ nach $(U_2,\Phi_2)$}.
\end{definition}

\begin{definition}
  Seien $M,f$ Mannigfaltigkeiten und $X$ eine Unterhalbgruppe von
  $\Diff(F)$. Sei weiterhin $\set{U_i}_{i\in I}$ eine offene
  Überdeckung von $M$ und $z = \set{ g_{ij} \colon U_i\cap U_j \to X}$
  eine Familie von Abbildungen. Man nennt $z$ einen
  \emph{$1$-Kozykel mit Werten in $X$}, falls er die \emph{Kozykelbedingung}
  \begin{statements}
  \item $g_{ii} = \id$
  \item $g_{ij} g_{jk} = g_{ik}$ auf $U_i\cap U_j\cap U_k$
  \end{statements}
  erfüllt.

  Eine Familie von Abbildungen $\set{h_i\colon U_i \to X}$ heißt
  \emph{$1$-Korand mit Werten in $X$}, falls die $h_i$ invertierbar
  sind. Zwei Kozykel $\set{g_{ij}}$ und
  $\set{\tilde{g}_{ij}}$ heißen \emph{kohomolog}, falls es einen
  Korand $\set{h_i}$ gibt, sodass $\tilde{g}_{ij} = h_i g_{ij} h_j^{-1}$ 
\end{definition}

\begin{proposition}
  Sei $\set{(U_i,\Phi_i)}_{i\in I}$ ein Bündelatlas des $F$-Bündels
  $\pi\colon E \to M$. Dann bilden die Bündelübergänge $g_{ij}$ einen
  $1$-Kozykel mit Werten in $\Diff(F)$.
\end{proposition}

\begin{definition}
  Für $i=1,2$ sei $\pi_i\colon E_i \to M$ ein $F$-Faserbündel. Ein
  \emph{Bündelhomomorphismus} 
  ist eine glatte Abbildung $\Psi\colon E_1 \to E_2$, sodass
  $\pi_2\circ F = \pi_1$. Ein \emph{Bündelisomorphismus} ist ein
  Diffeomorphismus $\Psi\colon E_1\to E_2$, sodass $\Psi$ und
  $\Psi^{-1}$ Bündelhomomorphismen sind. Zwei Bündel heißen
  \emph{isomorph}, falls es einen Bündelisomorphismus zwischen ihnen gibt.
\end{definition}

\begin{proposition}
  Seien $E,\tilde E \to M$ $F$-Faserbündel die beide über der offenen
  Überdeckung $\set{U_i}_{i\in I}$ trivialisieren mit Bündelatlanten
  $\set{(U_i,\Phi_i)}$ und $\set{(U_i,\tilde \Phi_i)}$. Dann sind die
  zugehörigen Übergangskozykel $\set{g_{ij}}$ und
  $\set{\tilde{g}_{ij}}$ genau dann kohomolog, wenn die Bündel
  isomorph sind.
\end{proposition}

\begin{satz}
  Seien $M,F$ Mannigfaltigkeiten und $\set{U_i}_{i\in I}$ eine offene
  Überdeckung von $M$. Ist $\set{g_{ij}}$ ein 1-Kozykel mit Werten in
  $\Diff(F)$, dann gibt ein bis auf Bündelisomorphie eindeutiges
  $F$-Faserbündel über $M$, dessen Übergangskozykel kohomolog zu
  $\set{g_{ij}}$ ist.
\end{satz}

\begin{definition}
  Sei $\pi\colon E \to M$ ein Faserbündel. Eine glatte Abbildung $s \colon M \to E$
  heißt \emph{glatter Schnitt}, falls $\pi\circ s = \id$. Wir
  bezeichnen
  \begin{equation*}
    \Gamma(E) := \set{s\colon M \to E\mid s\text{ glatter Schnitt}}
  \end{equation*}
\end{definition}

%%% Local Variables: 
%%% mode: latex
%%% TeX-master: "main"
%%% End: 

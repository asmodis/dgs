\section{Der Tangentialraum und das Differential an einen
  Punkt}
\label{sec:tpm}

\begin{definition}
  Sei $M$ eine Mannigfaltigkeit und $p\in M$. Wir definieren für $f,g
  \in C^\infty(M)$ die Relation
  \begin{equation*}
    f \sim_p g \iff \exists U\text{ Umgebung von $p$ mit } f|_U = g|_U
  \end{equation*}
  und bezeichnen mit $f_p := \set{g\in C^\infty(M) \mid g\sim_p f}$
  den sogenannten \emph{glatten Funktionskeim von $f$ in $p$} und mit
  $C^\infty(p)$ die Menge der glatten Funktionskeime in $p$.
\end{definition}

\begin{proposition}
  Sei $M$ eine Mannigfaltigkeit und $p\in M$. Dann gilt
  \begin{statements}
  \item $\sim_p$ ist eine Äquivalenzrelation.
  \item $C^\infty(p)$ ist eine kommutative $\R$-Algebra.
  \item Die Abbildung $C^\infty(p) \ni f_p \mapsto f(p) \in \R$ ist
    ein wohldefinierer $\R$-Algebrahomomorphismus. 
  \end{statements}
\end{proposition}

\begin{definition}
  Seien $A,B$ $\R$-Algebren und $\rho \colon A \to B$ ein
  Algebrahomomorphismus. Eine lineare Abbildung $\xi \colon A \to B$
  heißt \emph{lineare Derivation entlang $\rho$}, falls
  \begin{equation*}
    \xi(ab) = \xi(a)\rho(b) + \rho(a)\xi(b)
  \end{equation*}
  gilt.
\end{definition}

\begin{definition}
  Sei $M$ eine Mannigfaltigkeit und $p\in M$. Dann ist der
  \emph{Tangentialraum an $p$} definiert als
  \begin{equation*}
    T_pM := \Der_p(M) := \set{ \xi \colon C^{infty}(p) \to \R\mid
      \xi\text{ ist eine lineare Derivation entlang $f_p \mapsto f(p)$}}
  \end{equation*}
  Ein $\xi \in T_pM$ heißt \emph{Tangentialvektor}.

  Ist $f_p\in C^\infty(p)$ so heißt $\xi(f_p)$ die
  \emph{Richtungsableitung von $f$ in Richtung $\xi$}.
\end{definition}

\begin{proposition}
  Sei $M$ eine Mannigfaltigkeit, $p\in M$, $U$ eine Umgebung von $p$
  und $f\in C^\infty(U)$. Sei weiterhin $h$ eine Hutfunktion mit $p
  \in \supp h \subset U$. Für $\xi \in T_pM$ ist dann $\xi(f) := \xi((hf)_p)$
  wohldefiniert, das heißt unabhängig von $h$.

  \textsc{Notation:} Ist $f\in C^\infty(U)$ so bezeichne $f_p$ den
  Keim einer beliebigen Fortsetzung von $f$ auf $M$. Nach dieser
  Proposition ist dies wohldefiniert. Für $\xi \in T_pM$ bezeichne $\xi(f) := \xi(f_p)$.
\end{proposition}

\begin{definition}
  Sei $M$ eine Mannigfaltigkeit, $p\in M$ und $(U,x)$ eine Karte um
  $p$. Dann heißt
  \begin{equation*}
    \eval \deldelxi \at{p}\colon C^\infty(p) \to \R, f_p \mapsto \eval
    \deldelxi f\circ x^{-1}\at{x(p)}
  \end{equation*}
  der \emph{$i$-te Koordinatenvektor} für $i = 1,\dots,\dim M$.
\end{definition}

\begin{proposition}
  Sei $M$ eine Mannigfaltigkeit, $p\in M$ und $\xi \in T_pM$. Ist
  $c\colon p \mapsto c \in C^\infty(M)$ eine konstante Abbildung, so
  gilt $\xi(c) = 0$.

  Sind $f,g\in C^\infty(M)$ mit $f(p) = g(p) = 0$, so gilt $\xi(fg) = 0$.
\end{proposition}

\begin{lemma}
  Sei $U\subset \R^n$ ein Sterngebiet mit Mittelpunkt $x_0$ und
  $f\colon U \to \R$ sei $C^\infty$. Dann gibt es glatte Abbildungen
  $g_i \colon U \to \R$, sodass
  \begin{equation*}
    f(x) = f(x_0) + \sum_{i=1}^n g_i(x) \pr^i(x)
  \end{equation*}
  für alle $x\in U$, wobei $\pr^i$ die kanonische Projektion auf die
  $i$-te Komponente ist.
\end{lemma}

\begin{proposition}
  Sei $M^m$ eine Mannigfaltigkeit, $p\in M$ und $(U,x)$ eine Karte um
  $p$. Dann bilden die Koordinatenvektoren $\kvecxi[1],\dots,\kvecxi[m]$
  eine Basis von $T_pM$. Insbesondere ist also $\dim T_pM = \dim M$.
\end{proposition}

\begin{proposition}
  Sei $M^m$ eine Mannigfaltigkeit und $(U,x),(V,y)$ zwei Karten um $p\in
  M$. Sei $\xi \in T_pM$ mit
  \begin{equation*}
    \xi = \sum_{i=1}^m \eta^i \kvecxi = \sum_{i=1}^m \mu^i \kvecyi
  \end{equation*}
  und sei $\eta = (\eta^1,\dots,\eta^m)$, $\mu =
  (\mu^1,\dots,\mu^m)$. Dann gilt
  \begin{equation*}
    \eta = \Jac_{y(p)}(x\circ y^{-1})\mu
  \end{equation*}
\end{proposition}

\begin{definition}
  Sei $M$ eine Mannigfaltigkeit und sei $I\subset \R$ ein offenes Intervall. Eine glatte
  Abbildung $\alpha \colon I \to M$ heißt \emph{glatter Weg (glatte
    Kurve)} und falls $0\in I$ und $\alpha(0) = p$, so heißt $\alpha$
  auch \emph{Kurve (Weg) durch $p$}.

  Man definiert den \emph{Tangentenvektor von $\alpha$ zur Zeit $s$} durch
  \begin{equation*}
    \dot\alpha(s)(f) := \eval\ddt f\circ\alpha(t)\at{t=s} 
  \end{equation*}
  für alle $f\in C^\infty(\alpha(s))$.
\end{definition}

\begin{proposition}
  Sei $M$ eine Mannigfaltigkeit und $\alpha \colon I \to M$ eine
  glatte Kurve. Dann gilt $\dot\alpha(s) \in T_{\alpha(s)}M$.
\end{proposition}

\begin{proposition}
  Sei $M$ eine Mannigfaltigkeit. Ist $M$ ein Vektorraum, so gibt es
  für alle $p\in M$
  einen kanonischen Isomorphismus zwischen $M$ und $T_pM$.
\end{proposition}

\subsection{Das Differential einer Abbildung in einem Punkt}
\label{sec:differential}

\begin{definition}
  Seien $M,N$ Mannigfaltigkeiten, $f\colon M \to N$ glatt in $p\in
  M$. Das \emph{(totale) Differential von $f$ in $p$} ist
  \begin{equation*}
    d_pf \colon T_pM \to T_{f(p)}N, \xi \mapsto d_pf(\xi)
  \end{equation*}
  wobei $d_pf(\xi)$ für $g\in C^\infty(N)$ wie folgt definiert ist
  \begin{equation*}
    d_pf(\xi)(g) := \xi(g\circ f)
  \end{equation*}
\end{definition}

\begin{proposition}
  Seien $M,N$ Mannigfaltigkeiten, $f\colon M \to N$ glatt in $p\in
  M$. Dann ist das Differential $d_pf$ eine wohldefinierte lineare Abbildung.
\end{proposition}

\begin{definition}
  Seien $M,N$ Mannigfaltigkeiten, $f\colon M \to N$ glatt in $p\in
  M$. Dann ist der \emph{Rang von $f$ in $p$}, $\rnk_p f$, wie folgt definiert
  \begin{equation*}
    \rnk_p f := \dim d_pf(T_pM),
  \end{equation*}
  also als Rang der linearen Abbildung $d_pf$. Man sagt, $f$ hat
  \emph{vollen Rang in $p$}, falls $\rnk_p f = \min(\dim M,\dim N)$.

  Hat die Abbildung $f$ vollen Rang, so heißt sie
  \begin{statements}
  \item \emph{Submersion in $p$}, falls  $\dim M \geq \dim N$,
  \item \emph{Immersion in $p$}, falls $\dim M \leq \dim N$ und
  \item \emph{Bimersion in $p$}, falls $\dim M = \dim N$.
  \end{statements}

  Eine glatte Abbildung $f\colon M \to N$ heißt \emph{Einbettung},
  falls $f$ eine injektive Immersion ist und $f\colon M \to f(M)$ ein
  Homeomorphismus ist wenn man $f(M)$ mit der Teilraumtopologie versieht.
\end{definition}

\begin{proposition}
  Seien $M,N$ Mannigfaltigkeiten, $f\colon M \to N$ glatt in $p\in
  M$. Dann ist $f$ genau dann eine
  \begin{statements}
  \item Submersion, falls $d_pf$ surjektiv ist.
  \item Immersion, falls $d_pf$ injektiv ist.
  \item Bimersion, falls $d_pf$ bijektiv ist.
  \end{statements}
\end{proposition}

\begin{proposition}
  Seien $M,N$ Mannigfaltigkeiten, $f\colon M \to N$ glatt. Dann ist
  $\rnk f$ unterhalbsstetig, das heißt ist $\rnk_p f = k$, so gibt es
  eine Umgebung $U$ von $p$, sodass $\rnk_q f \geq k$ für alle $q\in
  U$.

  Insbesondere haben Submersionen, Immersionen und Bimersionen einen
  lokal konstanten Rang.
\end{proposition}

\begin{satz}[Satz vom konstanten Rang]
  Seien $M^m,N^n$ Mannigfaltigkeiten, $f\colon M \to N$ glatt. Sei $p\in
  M$ und $f$ habe lokal um $p$ konstanten Rang, das heißt es gibt eine
  Umgebung $U$ um $p$, sodass $\rnk_qf = \rnk_pf$ für alle $q\in
  U$. Dann gibt es eine Karte $(U,x=(x^1,\dots,x^m))$ um $p$ und eine
  Karte $(V,y=(y^1,\dots,y^n))$ um $f(p)$, sodass $y\circ f = (x^1,\dots,x^k,0,\dots,0)$.
\end{satz}

\begin{korollar}[Normalformensatz]
  Seien $M^m,N^n$ Mannigfaltigkeiten, $f\colon M \to N$ glatt. Dann gibt
  es genau dann eine Karte $(U,x)$ um $p\in M$ und eine Karte $(V,y)$ um $f(p)$, sodass
  \begin{statements}
  \item $y\circ f\circ x^{-1} = \pr$, falls $f$ eine Submersion (und
    damit $m\geq n$) ist
    (wobei $\pr$ die Projektion $\R^m\to\R^n, (x^1,\dots,x^m) \mapsto
    (x^1,\dots,x^n)$ bezeichnet).
  \item $y\circ f\circ x^{-1} = \incl$, falls $f$ eine Immersion (und
    damit $m\leq n$) ist
    (wobei $\incl$ die Projektion $\R^m\to\R^n, (x^1,\dots,x^m) \mapsto
    (x^1,\dots,x^m,0,\dots,0)$ bezeichnet).
  \item $y\circ f\circ x^{-1} = \id$, falls $f$ eine Bimersion (und
    damit $m = n$) ist.
  \end{statements}
\end{korollar}

\begin{korollar}[Umkehrsatz]
  Seien $M,N$ Mannigfaltigkeiten, $f\colon M \to N$ glatt. Dann ist
  $f$ genau dann eine Bimersion in $p$, falls $f$ ein lokaler
  Diffeomorphismus um $p$ ist, das heißt es gibt eine Umgebung von
  $p$, sodass $f|_U\colon U \to f(U)$ ein Diffeomorphismus ist. 
\end{korollar}

\begin{proposition}
  Seien $M,N$ Mannigfaltigkeiten, $f\colon M \to N$ glatt mit vollem
  Rang. Ist $f$ injektiv, so ist $f$ eine Immersion.
\end{proposition}

\begin{proposition}
  Seien $M,N$ Mannigfaltigkeiten und $f\colon M \to N$ eine
  Submersion. Dann ist $f$ offen, das heißt für jede offene Menge
  $U\subseteq M$ gilt, dass $f(U) \subseteq N$ offen ist.
\end{proposition}

\begin{proposition}
  Seien $M,N,P$ Mannigfaltigkeiten und $f\colon M\to N$, $g\colon N
  \to P$ glatte Abbildungen. Dann ist $d_p g\circ f = d_{f(p)}g \circ d_pf$.
\end{proposition}

\begin{proposition}
  Seien $M,N$ Mannigfaltigkeiten, $M$ zusammenhängend und $f\colon M
  \to N$ glatt. Gilt für
  alle $p\in M$, dass $d_pf = 0$, so ist $f$ konstant. 
\end{proposition}

%%% Local Variables: 
%%% mode: latex
%%% TeX-master: "main"
%%% End: 

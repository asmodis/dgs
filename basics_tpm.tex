\section{Der Tangentialraum an einen Punkt}
\label{sec:tpm}

\begin{definition}
  Sei $M$ eine Mannigfaltigkeit und $p\in M$. Wir definieren für $f,g
  \in C^\infty(M)$ die Relation
  \begin{equation*}
    f \sim_p g \iff \exists U\text{ Umgebung von $p$ mit } f|_U = g|_U
  \end{equation*}
  und bezeichnen mit $f_p := \set{g\in C^\infty(M) \mid g\sim_p f}$
  den sogenannten \emph{glatten Funktionskeim von $f$ in $p$} und mit
  $C^\infty(p)$ die Menge der glatten Funktionskeime in $p$.
\end{definition}

\begin{proposition}
  Sei $M$ eine Mannigfaltigkeit und $p\in M$. Dann gilt
  \begin{statements}
  \item $\sim_p$ ist eine Äquivalenzrelation.
  \item $C^\infty(p)$ ist eine kommutative $\R$-Algebra.
  \item Die Abbildung $C^\infty(p) \ni f_p \mapsto f(p) \in \R$ ist
    ein wohldefinierer $\R$-Algebrahomomorphismus. 
  \end{statements}
\end{proposition}

\begin{definition}
  Seien $A,B$ $\R$-Algebren und $\rho \colon A \to B$ ein
  Algebrahomomorphismus. Eine lineare Abbildung $\xi \colon A \to B$
  heißt \emph{lineare Derivation entlang $\rho$}, falls
  \begin{equation*}
    \xi(ab) = \xi(a)\rho(b) + \rho(a)\xi(b)
  \end{equation*}
  gilt.
\end{definition}

\begin{definition}
  Sei $M$ eine Mannigfaltigkeit und $p\in M$. Dann ist der
  \emph{Tangentialraum an $p$} definiert als
  \begin{equation*}
    T_pM := \Der_p(M) := \set{ \xi \colon C^{infty}(p) \to \R\mid
      \xi\text{ ist eine lineare Derivation entlang $f_p \mapsto f(p)$}}
  \end{equation*}
  Ein $\xi \in T_pM$ heißt \emph{Tangentialvektor}.

  Ist $f_p\in C^\infty(p)$ so heißt $\xi(f_p)$ die
  \emph{Richtungsableitung von $f$ in Richtung $\xi$}.
\end{definition}

\begin{proposition}
  Sei $M$ eine Mannigfaltigkeit, $p\in M$, $U$ eine Umgebung von $p$
  und $f\in C^\infty(U)$. Sei weiterhin $h$ eine Hutfunktion mit $p
  \in \supp h \subset U$. Für $\xi \in T_pM$ ist dann $\xi(f) := \xi((hf)_p)$
  wohldefiniert, das heißt unabhängig von $h$.

  \textsc{Notation:} Ist $f\in C^\infty(U)$ so bezeichne $f_p$ den
  Keim einer beliebigen Fortsetzung von $f$ auf $M$. Nach dieser
  Proposition ist dies wohldefiniert. Für $\xi \in T_pM$ bezeichne $\xi(f) := \xi(f_p)$.
\end{proposition}

\begin{definition}
  Sei $M$ eine Mannigfaltigkeit, $p\in M$ und $(U,x)$ eine Karte um
  $p$. Dann heißt
  \begin{equation*}
    \eval \deldelxi \at{p}\colon C^\infty(p) \to \R, f_p \mapsto \eval
    \deldelxi f\circ x^{-1}\at{x(p)}
  \end{equation*}
  der \emph{$i$-te Koordinatenvektor} für $i = 1,\dots,\dim M$.
\end{definition}

\begin{proposition}
  Sei $M$ eine Mannigfaltigkeit, $p\in M$ und $\xi \in T_pM$. Ist
  $c\colon p \mapsto c \in C^\infty(M)$ eine konstante Abbildung, so
  gilt $\xi(c) = 0$.

  Sind $f,g\in C^\infty(M)$ mit $f(p) = g(p) = 0$, so gilt $\xi(fg) = 0$.
\end{proposition}

\begin{lemma}
  Sei $U\subset \R^n$ ein Sterngebiet mit Mittelpunkt $x_0$ und
  $f\colon U \to \R$ sei $C^\infty$. Dann gibt es glatte Abbildungen
  $g_i \colon U \to \R$, sodass
  \begin{equation*}
    f(x) = f(x_0) + \sum_{i=1}^n g_i(x) \pr^i(x)
  \end{equation*}
  für alle $x\in U$, wobei $\pr^i$ die kanonische Projektion auf die
  $i$-te Komponente ist.
\end{lemma}

\begin{proposition}
  Sei $M^m$ eine Mannigfaltigkeit, $p\in M$ und $(U,x)$ eine Karte um
  $p$. Dann bilden die Koordinatenvektoren $\kvecxi[1],\dots,\kvecxi[m]$
  eine Basis von $T_pM$. Insbesondere ist also $\dim T_pM = \dim M$.
\end{proposition}

\begin{proposition}
  Sei $M^m$ eine Mannigfaltigkeit und $(U,x),(V,y)$ zwei Karten um $p\in
  M$. Sei $\xi \in T_pM$ mit
  \begin{equation*}
    \xi = \sum_{i=1}^m \eta^i \kvecxi = \sum_{i=1}^m \mu^i \kvecyi
  \end{equation*}
  und sei $\eta = (\eta^1,\dots,\eta^m)$, $\mu =
  (\mu^1,\dots,\mu^m)$. Dann gilt
  \begin{equation*}
    \eta = \Jac_{y(p)}(x\circ y^{-1})\mu
  \end{equation*}
\end{proposition}

\begin{definition}
  Sei $M$ eine Mannigfaltigkeit und sei $I\subset \R$ ein offenes Intervall. Eine glatte
  Abbildung $\alpha \colon I \to M$ heißt \emph{glatter Weg (glatte
    Kurve)} und falls $0\in I$ und $\alpha(0) = p$, so heißt $\alpha$
  auch \emph{Kurve (Weg) durch $p$}.

  Man definiert den \emph{Tangentenvektor von $\alpha$ zur Zeit $s$} durch
  \begin{equation*}
    \dot\alpha(s)(f) := \eval\ddt f\circ\alpha(t)\at{t=s} 
  \end{equation*}
  für alle $f\in C^\infty(\alpha(s))$.
\end{definition}

\begin{proposition}
  Sei $M$ eine Mannigfaltigkeit und $\alpha \colon I \to M$ eine
  glatte Kurve. Dann gilt $\dot\alpha(s) \in T_{\alpha(s)}M$.
\end{proposition}

\begin{proposition}
  Sei $M$ eine Mannigfaltigkeit. Ist $M$ ein Vektorraum, so gibt es
  für alle $p\in M$
  einen kanonischen Isomorphismus zwischen $M$ und $T_pM$.
\end{proposition}

%%% Local Variables: 
%%% mode: latex
%%% TeX-master: "main"
%%% End: 
